%
% Latex comments start with the percent symbol.
%
% This file should create a pdf on a mac or Linux command line by running:
%     pdflatex hw1.tex
% I usually add a few options
%     pdflatex -halt-on-error -interaction=nonstopmode -file-line-error hw1.tex
% 
% If you are new to Latex, you might not know that you may need to run the above
% twice for the compiler to sort out its references. (There are ways to finesse
% this). 
%

\documentclass[12pt]{report}

% Whether or not you need all these packages, or even some more will vary. These
% are some common ones, but not all are needed for this document. There is no
% real harm loading your favorites out of habit. 


\usepackage{algorithm,algorithmic,alltt,amsmath,amssymb,bm,
    cancel,color,fullpage,graphicx,listings,mathrsfs,
    multirow,setspace,subcaption,upgreek,xcolor}
\usepackage[numbered,framed]{matlab-prettifier}
\usepackage[colorlinks]{hyperref}
\usepackage[nameinlink,noabbrev]{cleveref}
\usepackage[verbose]{placeins}
\usepackage{caption}
\usepackage[skip=0.1ex, belowskip=1ex,
            labelformat=brace,
            singlelinecheck=off]{subcaption}

\doublespacing

%%%%%%%%%%%%%%%%%%%%%%%%%%%% Operators %%%%%%%%%%%%%%%%%%%%%%%%%%%%%%%
% Your personal shortcuts. You do not need to use any. argmax and argmin
\DeclareMathOperator*{\argmax}{arg\,max}
\DeclareMathOperator*{\argmin}{arg\,min}

%% Distributions
\newcommand{\N}{\mathcal{N}}
\newcommand{\U}{\mathcal{U}}
\newcommand{\Poi}{{\text Poisson}}
\newcommand{\Exp}{{\text Exp}}
\newcommand{\G}{\mathcal{G}}
\newcommand{\Ber}{{\text Bern}}
\newcommand{\Lap}{{\text Laplace}}
\newcommand{\btheta}{\boldsymbol{\theta}}
\newcommand{\bSigma}{\boldsymbol{\Sigma}}
% \usepackage[left=1cm,right=2.5cm,top=2cm,bottom=1.5cm]{geometry}
% Code blocks formatting

\definecolor{MyDarkGreen}{rgb}{0.0,0.4,0.0}

% For faster processing, load Matlab syntax for listings
\lstloadlanguages{Matlab}%
\lstset{language=Matlab,        % Use MATLAB
        % frame=single,   % Single frame around code
        basicstyle=\small\ttfamily,     % Use small true type font
        keywordstyle=[1]\color{blue}\bfseries,  % MATLAB functions bold and blue
        keywordstyle=[2]\color{purple}, % MATLAB function arguments purple
        keywordstyle=[3]\color{blue}\underbar,  % User functions underlined and blue
        identifierstyle=,       % Nothing special about identifiers
        % Comments small dark green courier
        commentstyle=\usefont{T1}{pcr}{m}{sl}\color{MyDarkGreen}\small,
        stringstyle=\color{purple},     % Strings are purple
        showstringspaces=false,         % Don't put marks in string spaces
        tabsize=2,      % 2 spaces per tab
        %%% Put standard MATLAB functions not included in the default language here
        morekeywords={xlim,ylim,var,alpha,factorial,poissrnd,normpdf,normcdf},
        %%% Put MATLAB function parameters here
        morekeywords=[2]{on, off, interp},
        %%% Put user defined functions here
        morekeywords=[3]{hw1,hw2,},
        gobble=4,
        morecomment=[l][\color{blue}]{...},     % Line continuation (...) like blue comment
        numbers=left,   % Line numbers on left
        firstnumber=1,          % Line numbers start with line 1
        numberstyle=\tiny\color{blue},  % Line numbers are blue
        stepnumber=5    % Line numbers go in steps of 5
}

%% Probability
\newcommand{\E}[1]{\mathbb{E}[#1]}
\newcommand{\Cov}[2]{\mathbb{C}\mathrm{ov}(#1,#2)}

%% Bold font for vectors from Ernesto, but I do not know how the first one
%  works, but it seems necessary for the second?
\def\*#1{\mathbf{#1}}
\newcommand*{\V}[1]{\mathbf{#1}}

%%%%%%%%%%%%%%%%%%%%%%%%%%%%%%%%%%%%%%%%%%%%%%%%%%%%%%%%%%%%%%%%%%%%%%

\begin{document}


\centerline{\it CS 577}
\centerline{\it HW \#5 Submission}
\centerline{\it Name: Joses Omojola}

Questions from Part A-D were completed in matlab and saved in the \emph{hw5.m} program. The program creates an \emph{output} folder to save images, 
so that the root directory is not always cluttered. The program can be run using \textit{hw5()}, and the results to coding questions are printed in 
terminal. File paths to input images are hardcoded, replace "IMG\_0862.png" with "IMG\_0861.jpeg" on Line 70 of \emph{hw5.m} if you get errors.

\begin{enumerate}

    \item[Part-A.]
    \ \\
    The world and image coordinates from \emph{hw4} were read into matlab and inverted using the homogenous least squares method to get the camera 
    matrix \textbf{M}. The inverted camera matrix 
    \[
    M = 
    \begin{bmatrix}
    0.0313 & 0.0532 & -0.1365 & 0.4854 \\
    -0.1393 & 0.0490 & -0.0236 & 0.8483 \\
    -0.0000 & -0.0000 & -0.0000 & 0.0009
    \end{bmatrix}
    \]

    The vizualization comparing the projected and original image coordinates is shown in \autoref{fig:Figure1}.\\
    The resulting \textbf{RMS error} between the projected points from "M" versus the original image coordinates was \textbf{9.5768}. This is lower than 
    the 17.01 and 41.28 RMS values that were obtained with camera matrix 1 and 2 in \emph{hw4}.  
    
    The calibration process does \textbf{not} minimize the sum of the squared errors(i.e., the sum of squared differences between observed and projected 
    points in the image plane). Homogeneous least squares minimizes an \textbf{algebraic error} in a linear system, not the reprojection error in image pixels. 
    $$P \cdot m = 0$$
    where $m$ represents the flattened camera matrix $P$. The solution to this minimizes the error in a linear sense (in terms of matrix $P$), which 
    corresponds to the smallest singular value of the matrix in the SVD decomposition. The projection from 3D to 2D is nonlinear, and the algebraic error 
    doesn't account for this transformation. While homogeneous least squares is computationally efficient and useful for getting an initial estimate of 
    the camera matrix, it is less than ideal for achieving accurate 2D projections. An ideal approach refines the initial matrix by minimizing reprojection 
    error through nonlinear optimization.

    \begin{figure}[H]
        \centering
        \includegraphics[scale=0.25]{output/f1_predicted_points_HLSQR.png}
        \caption{Visualization of projected points from "M" matrix.}
        \label{fig:Figure1}
    \end{figure}

    \FloatBarrier 

    \item[Part-B.]
    \ \\
    \begin{enumerate}
        \item[1.] Parametric equations for a sphere were used to generate 3D points at (x,y,z). The number of points were manually modified from 10 to 
        100, till there were no holes on the sphere.
        \item[2.] The camera position was assumed to be $[9, 14, 11]$.
        \item[3.] The dot products between the camera position and the outward normal vector of each point on the sphere was used to create a visibility 
        mask. Points less than zero are not visible, and were masked out.
        \item[4.] The Lambertian reflectance was calculated from the dot product between the sphere's surface normal and the normalized light direction. 
        Negative values were set to zero (self-shadow).
        \item[5.] The visible points were projected onto the 2D calibration image, using the camera matrix from \emph{Part A}. The points are shaded by 
        the lambertian reflectance values.
    \end{enumerate}
    The resulting plot of the projected sphere shaded by lambertian reflectance is shown in \autoref{fig:Figure2}. When the light source is in front of 
    the sphere, the reflectance is brightest in the increasing x-direction \autoref{fig:Figure2a}. However, when the light source is rotated behind the 
    sphere $[-30, 0, 0]$, the reflectance is highest in the decreasing x-direction facing the source \autoref{fig:Figure2b}. Considering that the reflectance 
    tracks with the light source, the resulting image makes sense.

    % \begin{figure}[H]
    %     \centering
    %     \includegraphics[scale=0.25]{output/f2_projected_sphere_with_visible_points.png}
    %     \caption{.}
    %     \label{fig:Figure2}
    % \end{figure}

    % \begin{figure}[H]
    %     \centering
    %     \includegraphics[scale=0.25]{output/f3_projected_sphere_with_rotated_light.png}
    %     \caption{Projected sphere like figure 2 with rotated light source.}
    %     \label{fig:Figure3}
    % \end{figure}

    \begin{figure}[!ht]\centering
        \hspace*{-1.2in}
        \begin{subfigure}{0.6\textwidth}
            \includegraphics[scale=0.23]{output/f2_projected_sphere_with_visible_points.png}
            \caption{Projected sphere with original light source at $[33, 29, 44]$.}
            \label{fig:Figure2a}
        \end{subfigure}
    \vfil
        \hspace*{-1.2in}
        \begin{subfigure}{0.6\textwidth}
        \includegraphics[scale=0.23]{output/f3_projected_sphere_with_rotated_light.png}
        \caption{Projected sphere with rotated light source at $[-30, 0, 0]$}
        \label{fig:Figure2b}
        \end{subfigure}
        \caption{Projected sphere at world coordinates $[3,2,3]$ using camera matrix from \emph{Part A}, 
        showing the effect of varying light sources. World coordinates are in inches.}
        \label{fig:Figure2}
    \end{figure}

    \FloatBarrier 


    % I read the tent figure from hw1 into matlab using the \emph{imread()} function, and used the \emph{imshow()} funtion to plot it with annotated 
    % labels. The annotations show increasing indices along both directions (\autoref{fig:Figure1}). The \emph{hw4.m} program prints out the figure 
    % dimensions (489,728,3) which indicates that the image matrix has 489 rows, 728 columns, and 3 channels.


    % The first index in the figure matrix corresponds to the y-axis in the figure, while the second index corresponds to the x-axis 
    % (\autoref{fig:Figure2}). The matrix index convention in the previous sentence is specific to matlab, and the number of rows of 
    % an image matrix are usually plotted on the y-axis of plots. The origin of the plot is in the top left corner, and the index 
    % for each axis increases from the plot origin. 

    % \begin{figure}[!ht]\centering
    %     \hspace*{-1.2in}
    %     \begin{subfigure}{0.40\textwidth}
    %         \includegraphics[scale=0.2]{output/tent_fig_x100_y300.png}
    %         \caption{Index location (x=100,y=300)}
    %         \label{fig:Figure2a}
    %     \end{subfigure}
    % \hfil
    %     \begin{subfigure}{0.40\textwidth}
    %     \includegraphics[scale=0.2]{output/tent_fig_bot_left.png}
    %     \caption{Bottom left index location}
    %     \label{fig:Figure2b}
    %     \end{subfigure}
    %     \caption{Index locations along tent figure}
    %     \label{fig:Figure2}
    % \end{figure}

    % \FloatBarrier 

    % The figure matrix coordinates are \textbf{not equivalent} to the plot coordinates. For a matrix A of size m x n, element 
    % $A(i, j)$ corresponds to row i and column j. The point at matrix location $A(i, j)$ gets plotted at the coordinate $(j, i)$ in 
    % the matlab figure, where: $j$ is the x-coordinate and $i$ is the y-coordinate. To convert figure indices to click indices,
    % ensure the figure rows are on the y-axis, and figure columns are on the x-axis. 

    % Pixel colors can be changed in matlab using the colon notation. E.g to change the pixel colors at location (100,200) to red,
    % \begin{lstlisting}[language=Matlab]
    % img(100, 200, :) = [255 0 0];
    % \end{lstlisting}
    % The resulting pixel change is shown in \autoref{fig:Figure3a}, and the zoomed bounding box around the altered pixel is shown 
    % in \autoref{fig:Figure3b}.


    % \begin{figure}[!ht]\centering
    %     \hspace*{-1.2in}
    %     \begin{subfigure}{0.40\textwidth}
    %         \includegraphics[scale=0.3]{output/f2_tent_red_pixed_100_200.png}
    %         \caption{Plot of red pixel in tent figure}
    %         \label{fig:Figure3a}
    %     \end{subfigure}
    % \hfil
    %     \begin{subfigure}{0.40\textwidth}
    %     \includegraphics[scale=0.25]{output/f2_zoomed.png}
    %     \caption{Zoomed pixel with boundary box}
    %     \label{fig:Figure3b}
    %     \end{subfigure}
    %     \caption{Effect of altering pixel data}
    %     \label{fig:Figure3}
    % \end{figure}

    % \FloatBarrier 

    % \item[Part-B.]
    % \ \\
    % Annotations were drawn on the provided figure to indicate the axis along the x,y, and z directions (\autoref{fig:Figure4}). Fifteen 
    % different points (\textit{5 along each axis}) were selected and used to create the \emph{world\_coords.txt} and \emph{image\_coords.txt} 
    % files.

    % \begin{figure}[H]
    %     \centering
    %     \includegraphics[scale=0.25]{output/f3_selected_points.png}
    %     \caption{Calibration figure with labeled axis and selected calibration points (white dots).}
    %     \label{fig:Figure4}
    % \end{figure}

    % \FloatBarrier 

    % The text files for the image and world coordinates are in the submission packet.

    % \item[Part-C.]
    % \ \\
    % The camera matrices in the hw4 instructions were used to project the world coordinates from \textbf{Part B} to the image coordinates. The basic 
    % camera model relates 3D world coordinates \((X, Y, Z)\) to 2D image coordinates \((u, v)\) using the \textit{camera matrix} \(P\). The camera projection 
    % equation is:
    % \[
    % \begin{bmatrix}
    %     u \\
    %     v \\
    %     1
    % \end{bmatrix}
    % =
    % P \cdot
    % \begin{bmatrix}
    %     X \\
    %     Y \\
    %     Z \\
    %     1
    % \end{bmatrix}
    % \]

    % Where:  \\
    % - \(P\) is the 3x4 \textit{camera matrix}.\\
    % - \((X, Y, Z)\) are the \textit{real-world coordinates}.\\
    % - \((u, v)\) are the \textit{image coordinates} in pixels.\\
    
    % The click coordinates were swapped to orient them with the indexing coordinate system prior to estimation of RMS errors. The reprojected points from 
    % both matrices is shown in \autoref{fig:Figure5} below.

    % \begin{figure}[H]
    %     \centering
    %     \includegraphics[scale=0.25]{output/f4_reprojected_points.png}
    %     \caption{Calibration figure with overlain calibration and reprojected points.}
    %     \label{fig:Figure5}
    % \end{figure}

    % \FloatBarrier 

    % Overall, the points from the first matrix are closer to the ground truth calibration points than the second matrix. Camera matrix 1 has a RMS error of \textbf{17.01}, 
    % while camera matrix 2 has a RMS error of \textbf{41.28}. This indicates that the first matrix is more accurate. 


    % \item[Part-D1.]
    % \ \\
    % The inverse operation of a 3D translation in the direction (tx, ty, tz) is a translation in the opposite direction, (-tx, -ty, -tz).\\
    % \textbf{Proof}:\\
    % Consider a point P with coordinates (x, y, z). \\
    % After applying the translation (tx, ty, tz), the new coordinates of P are (x + tx, y + ty, z + tz). \\
    % Now, if we apply the translation (-tx, -ty, -tz) to the new coordinates, we get:
    % $$
    % (x + tx - tx, y + ty - ty, z + tz - tz) = (x, y, z)
    % $$
    % Thus, the second translation undoes the first, and we have indeed found the inverse operation.

    % \item[Part-D2.]
    % \ \\
    % To prove that $T_1 T_2 = T_2 T_1$, we consider a translation matrix in 3D which has the following form:

    % \[
    % T(t_x, t_y, t_z) = 
    % \begin{bmatrix}
    % 1 & 0 & 0 & t_x \\
    % 0 & 1 & 0 & t_y \\
    % 0 & 0 & 1 & t_z \\
    % 0 & 0 & 0 & 1
    % \end{bmatrix}
    % \]

    % Where \(t_x\), \(t_y\), and \(t_z\) are the translation components along the \(x\)-, \(y\)-, and \(z\)-axes, respectively. \\
    % We can define two translation matrices, \(T_1\) and \(T_2\), corresponding to translations by \((t_{x1}, t_{y1}, t_{z1})\) and 
    % \((t_{x2}, t_{y2}, t_{z2})\), respectively.

    % $$
    % T_1 =
    % \begin{bmatrix}
    % 1 & 0 & 0 & t_{x1} \\
    % 0 & 1 & 0 & t_{y1} \\
    % 0 & 0 & 1 & t_{z1} \\
    % 0 & 0 & 0 & 1
    % \end{bmatrix},
    % T_2 =
    % \begin{bmatrix}
    % 1 & 0 & 0 & t_{x2} \\
    % 0 & 1 & 0 & t_{y2} \\
    % 0 & 0 & 1 & t_{z2} \\
    % 0 & 0 & 0 & 1
    % \end{bmatrix}
    % $$

    % The product \(T_1 T_2\) is:
    % $$
    % T_1 T_2 =
    % \begin{bmatrix}
    % 1 & 0 & 0 & t_{x1} \\
    % 0 & 1 & 0 & t_{y1} \\
    % 0 & 0 & 1 & t_{z1} \\
    % 0 & 0 & 0 & 1
    % \end{bmatrix}
    % \begin{bmatrix}
    % 1 & 0 & 0 & t_{x2} \\
    % 0 & 1 & 0 & t_{y2} \\
    % 0 & 0 & 1 & t_{z2} \\
    % 0 & 0 & 0 & 1
    % \end{bmatrix}
    % =
    % \begin{bmatrix}
    % 1 & 0 & 0 & (t_{x1} + t_{x2}) \\
    % 0 & 1 & 0 & (t_{y1} + t_{y2}) \\
    % 0 & 0 & 1 & (t_{z1} + t_{z2}) \\
    % 0 & 0 & 0 & 1
    % \end{bmatrix}
    % $$

    % The product \(T_2 T_1\) is:

    % $$
    % T_2 T_1 =
    % \begin{bmatrix}
    % 1 & 0 & 0 & t_{x2} \\
    % 0 & 1 & 0 & t_{y2} \\
    % 0 & 0 & 1 & t_{z2} \\
    % 0 & 0 & 0 & 1
    % \end{bmatrix}
    % \begin{bmatrix}
    % 1 & 0 & 0 & t_{x1} \\
    % 0 & 1 & 0 & t_{y1} \\
    % 0 & 0 & 1 & t_{z1} \\
    % 0 & 0 & 0 & 1
    % \end{bmatrix}
    % =
    % \begin{bmatrix}
    % 1 & 0 & 0 & (t_{x2} + t_{x1}) \\
    % 0 & 1 & 0 & (t_{y2} + t_{y1}) \\
    % 0 & 0 & 1 & (t_{z2} + t_{z1}) \\
    % 0 & 0 & 0 & 1
    % \end{bmatrix}
    % $$

    % which yields the same result, implying that \textbf{translation matrices are commutative}.

    % \item[Part-D3.]
    % \ \\
    % For points represented in homogeneous coordinates where the homogeneous component \(w\) is not necessarily 1, we consider a general case where the homogeneous 
    % coordinate \(w\) can take any non-zero value.

    % A point in 3D space \((x, y, z)\) is typically represented in homogeneous coordinates as \((x, y, z, w)\), where \(w \neq 0\). In homogeneous coordinates:
    % \[
    % \begin{bmatrix}
    % x' \\
    % y' \\
    % z' \\
    % w'
    % \end{bmatrix}
    % =
    % \lambda
    % \begin{bmatrix}
    % x \\
    % y \\
    % z \\
    % 1
    % \end{bmatrix}
    % \]
    % where \(\lambda = \frac{1}{w}\).\\
    % When \(w = 1\), the point is in Cartesian coordinates. When \(w \neq 1\), the point in Cartesian coordinates is recovered as:

    % \[
    % \left( \frac{x}{w}, \frac{y}{w}, \frac{z}{w} \right)
    % \]

    % The translation matrix is the same as \(T_1\) above. We can apply the translation matrix \(T(t_x, t_y, t_z)\) to a point represented by homogeneous coordinates
    %  \((x, y, z, w)\), where \(w \neq 1\) as:

    % \[
    % \begin{bmatrix}
    % x \\
    % y \\
    % z \\
    % w
    % \end{bmatrix}
    % \]

    % Multiplying this point by the translation matrix:

    % \[
    % T(t_x, t_y, t_z) 
    % \begin{bmatrix}
    % x \\
    % y \\
    % z \\
    % w
    % \end{bmatrix}
    % =
    % \begin{bmatrix}
    % 1 & 0 & 0 & t_x \\
    % 0 & 1 & 0 & t_y \\
    % 0 & 0 & 1 & t_z \\
    % 0 & 0 & 0 & 1
    % \end{bmatrix}
    % \begin{bmatrix}
    % x \\
    % y \\
    % z \\
    % w
    % \end{bmatrix}
    % =
    % \begin{bmatrix}
    % x + w t_x \\
    % y + w t_y \\
    % z + w t_z \\
    % w
    % \end{bmatrix}
    % \]
    
    % The result of applying the translation matrix to the homogeneous coordinate point \((x, y, z, w)\) is:

    % \[
    % \begin{bmatrix}
    % x' \\
    % y' \\
    % z' \\
    % w'
    % \end{bmatrix}
    % =
    % \begin{bmatrix}
    % x + w t_x \\
    % y + w t_y \\
    % z + w t_z \\
    % w
    % \end{bmatrix}
    % \]

    % Where: \\
    % - \(x' = x + w t_x\) \\
    % - \(y' = y + w t_y\) \\
    % - \(z' = z + w t_z\) \\
    % - \(w' = w\) (unchanged) \\
    % To convert this result back to Cartesian coordinates, we divide the first three coordinates by \(w\):

    % \[
    % \left( \frac{x'}{w}, \frac{y'}{w}, \frac{z'}{w} \right) =
    % \left( \frac{x + w t_x}{w}, \frac{y + w t_y}{w}, \frac{z + w t_z}{w} \right) =
    % \left( \frac{x}{w} + t_x, \frac{y}{w} + t_y, \frac{z}{w} + t_z \right)
    % \]

    % This is what we expect from a translation in Cartesian coordinates, where the point \(\left( \frac{x}{w}, \frac{y}{w}, \frac{z}{w} \right)\) 
    % (the original point in Cartesian coordinates) has been translated by \((t_x, t_y, t_z)\) even when \(w \neq 1\).

    % \item[Part-E.]  
    % \ \\
    % 1. \textbf{Scale}: The interval $[0,0] \to [1,0]$ and $[0,0] \to [0,1]$ both map to 400 pixels in the image. This indicates a uniform scale of 400 pixels per unit 
    % in both the x- and y-directions.\\
    % 2. \textbf{Translation}: The origin $(0,0)$ in the XY coordinate system maps to the pixel coordinate $(400, 600)$ in the image coordinate system.\\
    % 3. \textbf{Rotation/Flip}: There may be some rotation or flip involved to align the XY coordinate system with the image coordinate system. We'll assume that flipping 
    % along the y-axis is needed because image coordinates usually increase downwards, while typical XY coordinates increase upwards.

    % The translation matrix \(T\) is given by:
    % $$
    % T =
    % \begin{bmatrix}
    % 1 & 0 & 400 \\
    % 0 & 1 & 600 \\
    % 0 & 0 & 1
    % \end{bmatrix}
    % $$
    
    % The scaling matrix \(S\) is:

    % $$
    % S =
    % \begin{bmatrix}
    % 400 & 0 & 0 \\
    % 0 & 400 & 0 \\
    % 0 & 0 & 1
    % \end{bmatrix}
    % $$

    % To account for the typical flip in image coordinates (where the y-axis increases downwards), we need a flip along the y-axis. The flipping matrix \(F\) is:
    % $$
    % F =
    % \begin{bmatrix}
    % 1 & 0 & 0 \\
    % 0 & -1 & 0 \\
    % 0 & 0 & 1
    % \end{bmatrix}
    % $$
    % This matrix flips the y-values, reversing the direction of the y-axis.  

    % The final transformation matrix \(M\) that maps from the XY coordinate system to the image coordinate system is the product of 
    % the three matrices: \textit{scaling,flipping, and translation}:

    % $$
    % M = T \cdot F \cdot S = 
    % \begin{bmatrix}
    % 1 & 0 & 400 \\
    % 0 & 1 & 600 \\
    % 0 & 0 & 1
    % \end{bmatrix}
    % \begin{bmatrix}
    % 400 & 0 & 0 \\
    % 0 & -400 & 0 \\
    % 0 & 0 & 1
    % \end{bmatrix}
    % =
    % \begin{bmatrix}
    % 400 & 0 & 400 \\
    % 0 & -400 & 600 \\
    % 0 & 0 & 1
    % \end{bmatrix}
    % $$ \\

    
    % We can apply the transformation matrix \(M\) to the given points \((-0.5, -0.5)\), \((-0.5, 0.5)\), and \((0,1)\). Each point \((x, y)\) 
    % in the XY coordinate system is represented in homogeneous coordinates as \([x, y, 1]^T\).

    % \begin{enumerate}
    % \item[1.] Point \((-0.5, -0.5)\): \\
    % $$
    % M 
    % \begin{bmatrix}
    % -0.5 \\
    % -0.5 \\
    % 1
    % \end{bmatrix}
    % =
    % \begin{bmatrix}
    % 200 \\
    % 800 \\
    % 1
    % \end{bmatrix}
    % = (200, 800)
    % $$

    % \item[2.] Point \((-0.5, 0.5) = (200, 400)\)

    % \item[3.] Point \((0, 1) = (400, 200)\)

    % \item[4.] Additional Point 1: \((1, 0) = (800, 600)\)

    % \item[5.] Additional Point 2: \((0, -1) = (400, 1000)\)

    % \end{enumerate}

    % For the inverse mapping of \((1,1)\), we need to invert the transformation matrix \(M\). The inverse of \(M\) is:

    % $$
    % M^{-1} =
    % \begin{bmatrix}
    % \frac{1}{400} & 0 & -1 \\
    % 0 & \frac{-1}{400} & \frac{3}{2} \\
    % 0 & 0 & 1
    % \end{bmatrix}
    % $$
    
    % We can now apply this inverse to pixel \((1,1)\):

    % $$
    % M^{-1}
    % \begin{bmatrix}
    % 1 \\
    % 1 \\
    % 1
    % \end{bmatrix}
    % =
    % \begin{bmatrix}
    % -0.9975 \\
    % 1.4975 \\
    % 1
    % \end{bmatrix}
    % = (-0.9975, 1.4975)
    % $$

    % The mappings of the transformed points can be visualized in \autoref{fig:Figure6}. \textbf{Note} how the y-axis coordinates are 
    % reversed because of the flipping transformation.
    % \begin{figure}[H]
    %     \centering
    %     \includegraphics[scale=0.6]{output/f5_transformed_points.png}
    %     \caption{Illustration of transformed points.}
    %     \label{fig:Figure6}
    % \end{figure}

    % \FloatBarrier 

\end{enumerate}

\end{document}