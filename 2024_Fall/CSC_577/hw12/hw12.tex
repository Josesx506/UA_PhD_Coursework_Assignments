%
% Latex comments start with the percent symbol.
%
% This file should create a pdf on a mac or Linux command line by running:
%     pdflatex hw12.tex
% I usually add a few options
%     pdflatex -halt-on-error -interaction=nonstopmode -file-line-error hw12.tex
% 
% If you are new to Latex, you might not know that you may need to run the above
% twice for the compiler to sort out its references. (There are ways to finesse
% this). 
%

\documentclass[12pt]{report}

% Whether or not you need all these packages, or even some more will vary. These
% are some common ones, but not all are needed for this document. There is no
% real harm loading your favorites out of habit. 


\usepackage{algorithm,algorithmic,alltt,amsmath,amssymb,bm,
    cancel,color,fullpage,graphicx,listings,mathrsfs,
    multirow,setspace,subcaption,upgreek,xcolor}
\usepackage[numbered,framed]{matlab-prettifier}
\usepackage[colorlinks]{hyperref}
\usepackage[nameinlink,noabbrev]{cleveref}
\usepackage[verbose]{placeins}
\usepackage{caption}
\usepackage[skip=0.1ex, belowskip=1ex,
            labelformat=brace,
            singlelinecheck=off]{subcaption}
\usepackage{float, wrapfig, multicol}

\doublespacing

%%%%%%%%%%%%%%%%%%%%%%%%%%%% Operators %%%%%%%%%%%%%%%%%%%%%%%%%%%%%%%
% Your personal shortcuts. You do not need to use any. argmax and argmin
\DeclareMathOperator*{\argmax}{arg\,max}
\DeclareMathOperator*{\argmin}{arg\,min}

%% Distributions
\newcommand{\N}{\mathcal{N}}
\newcommand{\U}{\mathcal{U}}
\newcommand{\Poi}{{\text Poisson}}
\newcommand{\Exp}{{\text Exp}}
\newcommand{\G}{\mathcal{G}}
\newcommand{\Ber}{{\text Bern}}
\newcommand{\Lap}{{\text Laplace}}
\newcommand{\btheta}{\boldsymbol{\theta}}
\newcommand{\bSigma}{\boldsymbol{\Sigma}}
% \usepackage[left=1cm,right=2.5cm,top=2cm,bottom=1.5cm]{geometry}
% Code blocks formatting

\definecolor{MyDarkGreen}{rgb}{0.0,0.4,0.0}

% For faster processing, load Matlab syntax for listings
\lstloadlanguages{Matlab}%
\lstset{language=Matlab,        % Use MATLAB
        % frame=single,   % Single frame around code
        basicstyle=\small\ttfamily,     % Use small true type font
        keywordstyle=[1]\color{blue}\bfseries,  % MATLAB functions bold and blue
        keywordstyle=[2]\color{purple}, % MATLAB function arguments purple
        keywordstyle=[3]\color{blue}\underbar,  % User functions underlined and blue
        identifierstyle=,       % Nothing special about identifiers
        % Comments small dark green courier
        commentstyle=\usefont{T1}{pcr}{m}{sl}\color{MyDarkGreen}\small,
        stringstyle=\color{purple},     % Strings are purple
        showstringspaces=false,         % Don't put marks in string spaces
        tabsize=2,      % 2 spaces per tab
        %%% Put standard MATLAB functions not included in the default language here
        morekeywords={xlim,ylim,var,alpha,factorial,poissrnd,normpdf,normcdf},
        %%% Put MATLAB function parameters here
        morekeywords=[2]{on, off, interp},
        %%% Put user defined functions here
        morekeywords=[3]{hw1,hw2,},
        gobble=4,
        morecomment=[l][\color{blue}]{...},     % Line continuation (...) like blue comment
        numbers=left,   % Line numbers on left
        firstnumber=1,          % Line numbers start with line 1
        numberstyle=\tiny\color{blue},  % Line numbers are blue
        stepnumber=5    % Line numbers go in steps of 5
}

%% Probability
\newcommand{\E}[1]{\mathbb{E}[#1]}
\newcommand{\Cov}[2]{\mathbb{C}\mathrm{ov}(#1,#2)}

%% Bold font for vectors from Ernesto, but I do not know how the first one
%  works, but it seems necessary for the second?
\def\*#1{\mathbf{#1}}
\newcommand*{\V}[1]{\mathbf{#1}}

%%%%%%%%%%%%%%%%%%%%%%%%%%%%%%%%%%%%%%%%%%%%%%%%%%%%%%%%%%%%%%%%%%%%%%

\begin{document}


\centerline{\it CS 577}
\centerline{\it HW \#12 Submission}
\centerline{\it Name: Joses Omojola}

Questions from Part A - C were completed in matlab and saved in the \emph{hw12.m} program. The program creates an \emph{output} folder to save images, so that the 
root directory is not always cluttered, and it can be run using the \textit{hw12()} command. 

\begin{enumerate}

    \item[Part A.] The script reads in the Line2 data and fits a line to it (\autoref{fig:Figure1}). The resulting line equation was $0.221x + 0.423y + -0.879 = 0$, with 
    a perpendicular distance error of 9.5. 111 inlier points were identified out of a total of 300 points.

    \begin{figure}[H]
        \centering
        \includegraphics[scale=0.75]{output/f1_ransac.png}
        \caption{RANSAC fit on Line2 data using the threshold method after 100 iterations.}
        \label{fig:Figure1}
    \end{figure}

    \item[Part B.] Using the synthetic dataset for 10 samples between $(0,1)$, the homography transformation RMS error for 4, 5, and 6 points was 0.0, 0.45, and 0.79 
    respectively. This suggests that using more points can increase the transformation error, however, I noticed that including more points on the real data from the 
    image pairs improved the quality of my transformations, verifying that the implemented DLT method is working.  

    I manually selected 8 keypoints for the slide/frame pairs from \emph{hw9}. 4 of the selected points were used to compute the homography matrix, and the resulting matrix 
    was used to predict on the holdout points. The results are shown in \Cref{fig:Figure2,fig:Figure3,fig:Figure4}. The clicked keypoints are shown as purple circles on the 
    frame images (right). The original keypoints on the slide image and homography transformed points on the frame image are shown as red squares, connected by yellow lines.

    \begin{figure}[H]
        \centering
        \includegraphics[scale=0.65]{output/f2_dlt_homo_fg_1.png}
        \caption{Homography projections for image pair 1. Annotation descriptions are described in preceding paragraph.}
        \label{fig:Figure2}
    \end{figure}

    \begin{figure}[H]
        \centering
        \includegraphics[scale=0.65]{output/f3_dlt_homo_fg_2.png}
        \caption{Homography projections for image pair 2. Annotation descriptions are described in preceding paragraph.}
        \label{fig:Figure3}
    \end{figure}

    \begin{figure}[H]
        \centering
        \includegraphics[scale=0.65]{output/f4_dlt_homo_fg_3.png}
        \caption{Homography projections for image pair 3. Annotation descriptions are described in preceding paragraph.}
        \label{fig:Figure4}
    \end{figure}


    \item[Part C.]  RANSAC was used to improve keypoint matching. The source and destination keypoints were extracted from SIFT features, by computing the Euclidean distance with a 
    Lowe ratio cutoff of 0.8. Four random points were used to compute RANSAC for 100 iterations. Reprojection errors were computed from homography transformations, and inlier matches 
    are plotted in \Cref{fig:Figure5,fig:Figure6,fig:Figure7}. Comparing the reconstruction error with and without RANSAC shows that the RANSAC implementation improves the accuracy of 
    projected keypoints (This is printed in the matlab script). Image pairs 1 and 3 provided very good responses compared to image 2.

    \begin{figure}[H]
        \centering
        \includegraphics[scale=0.65]{output/f5_sift_dlt_homo_fg_1.png}
        \caption{Homography projections of SIFT keypoint features filtered by RANSAC inliers for image pair 1.}
        \label{fig:Figure5}
    \end{figure}

    \begin{figure}[H]
        \centering
        \includegraphics[scale=0.65]{output/f6_sift_dlt_homo_fg_2.png}
        \caption{Homography projections of SIFT keypoint features filtered by RANSAC inliers for image pair 2.}
        \label{fig:Figure6}
    \end{figure}

    \begin{figure}[H]
        \centering
        \includegraphics[scale=0.65]{output/f7_sift_dlt_homo_fg_3.png}
        \caption{Homography projections of SIFT keypoint features filtered by RANSAC inliers for image pair 3.}
        \label{fig:Figure7}
    \end{figure}


\end{enumerate}

\end{document}