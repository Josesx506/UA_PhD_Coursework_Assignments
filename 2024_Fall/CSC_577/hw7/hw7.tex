%
% Latex comments start with the percent symbol.
%
% This file should create a pdf on a mac or Linux command line by running:
%     pdflatex hw7.tex
% I usually add a few options
%     pdflatex -halt-on-error -interaction=nonstopmode -file-line-error hw7.tex
% 
% If you are new to Latex, you might not know that you may need to run the above
% twice for the compiler to sort out its references. (There are ways to finesse
% this). 
%

\documentclass[12pt]{report}

% Whether or not you need all these packages, or even some more will vary. These
% are some common ones, but not all are needed for this document. There is no
% real harm loading your favorites out of habit. 


\usepackage{algorithm,algorithmic,alltt,amsmath,amssymb,bm,
    cancel,color,fullpage,graphicx,listings,mathrsfs,
    multirow,setspace,subcaption,upgreek,xcolor}
\usepackage[numbered,framed]{matlab-prettifier}
\usepackage[colorlinks]{hyperref}
\usepackage[nameinlink,noabbrev]{cleveref}
\usepackage[verbose]{placeins}
\usepackage{caption}
\usepackage[skip=0.1ex, belowskip=1ex,
            labelformat=brace,
            singlelinecheck=off]{subcaption}

\doublespacing

%%%%%%%%%%%%%%%%%%%%%%%%%%%% Operators %%%%%%%%%%%%%%%%%%%%%%%%%%%%%%%
% Your personal shortcuts. You do not need to use any. argmax and argmin
\DeclareMathOperator*{\argmax}{arg\,max}
\DeclareMathOperator*{\argmin}{arg\,min}

%% Distributions
\newcommand{\N}{\mathcal{N}}
\newcommand{\U}{\mathcal{U}}
\newcommand{\Poi}{{\text Poisson}}
\newcommand{\Exp}{{\text Exp}}
\newcommand{\G}{\mathcal{G}}
\newcommand{\Ber}{{\text Bern}}
\newcommand{\Lap}{{\text Laplace}}
\newcommand{\btheta}{\boldsymbol{\theta}}
\newcommand{\bSigma}{\boldsymbol{\Sigma}}
% \usepackage[left=1cm,right=2.5cm,top=2cm,bottom=1.5cm]{geometry}
% Code blocks formatting

\definecolor{MyDarkGreen}{rgb}{0.0,0.4,0.0}

% For faster processing, load Matlab syntax for listings
\lstloadlanguages{Matlab}%
\lstset{language=Matlab,        % Use MATLAB
        % frame=single,   % Single frame around code
        basicstyle=\small\ttfamily,     % Use small true type font
        keywordstyle=[1]\color{blue}\bfseries,  % MATLAB functions bold and blue
        keywordstyle=[2]\color{purple}, % MATLAB function arguments purple
        keywordstyle=[3]\color{blue}\underbar,  % User functions underlined and blue
        identifierstyle=,       % Nothing special about identifiers
        % Comments small dark green courier
        commentstyle=\usefont{T1}{pcr}{m}{sl}\color{MyDarkGreen}\small,
        stringstyle=\color{purple},     % Strings are purple
        showstringspaces=false,         % Don't put marks in string spaces
        tabsize=2,      % 2 spaces per tab
        %%% Put standard MATLAB functions not included in the default language here
        morekeywords={xlim,ylim,var,alpha,factorial,poissrnd,normpdf,normcdf},
        %%% Put MATLAB function parameters here
        morekeywords=[2]{on, off, interp},
        %%% Put user defined functions here
        morekeywords=[3]{hw1,hw2,},
        gobble=4,
        morecomment=[l][\color{blue}]{...},     % Line continuation (...) like blue comment
        numbers=left,   % Line numbers on left
        firstnumber=1,          % Line numbers start with line 1
        numberstyle=\tiny\color{blue},  % Line numbers are blue
        stepnumber=5    % Line numbers go in steps of 5
}

%% Probability
\newcommand{\E}[1]{\mathbb{E}[#1]}
\newcommand{\Cov}[2]{\mathbb{C}\mathrm{ov}(#1,#2)}

%% Bold font for vectors from Ernesto, but I do not know how the first one
%  works, but it seems necessary for the second?
\def\*#1{\mathbf{#1}}
\newcommand*{\V}[1]{\mathbf{#1}}

%%%%%%%%%%%%%%%%%%%%%%%%%%%%%%%%%%%%%%%%%%%%%%%%%%%%%%%%%%%%%%%%%%%%%%

\begin{document}


\centerline{\it CS 577}
\centerline{\it HW \#7 Submission}
\centerline{\it Name: Joses Omojola}

Questions from Part A-B were completed in matlab and saved in the \emph{hw7.m} program. The program creates an \emph{output} folder to save images, so that the 
root directory is not always cluttered, and it can be run using the \textit{hw7()} command. This prints the results to coding questions within terminal. All numbers 
are rounded to \emph{3 s.f} for reporting purposes and actual values are more precise.

\begin{enumerate}

    \item[Q-A1.]
    \ \\
    The white light values from the macbeth image were gotten by plotting the macbeth image with \emph{imshow()} and \emph{datacursormode on}. The array limits 
    of the white pixels were selected using colon notation and averaged to get a white light color of $[238,220,250]$.
    
    \item[Q-A2.]
    \ \\
    The same workflow in QA1 was applied to the macbeth solux image (unknown light source) to get a white light color of $[132,159,250]$.
    
    \item[Q-A3.]
    \ \\
    The angular error was estimated by measuring the cosine angles between both color vectors in the previous questions, and the resulting value was $13.8^o$.
    
    \item[Q-A4.]
    \ \\
    The diagonal matrix was used to correct the solux image and the comparison between the solux, corrected, and canonical images are shown in \autoref{fig:Figure1}.

    \begin{figure}[H]
        \centering
        \includegraphics[scale=0.5]{output/f1_macbeth_img_results.png}
        \caption{Comparison of macbeth images. The bluish effect is removed from the corrected image in the middle panel. All images have been scaled to have a 
        max RGB pixel value of 250 using a single scale factor.}
        \label{fig:Figure1}
    \end{figure}

    \FloatBarrier 

    The derived diagonal matrix (“oracle” color constancy illuminant) was 
    $$
    D = 
    \begin{bmatrix}
    1.80 & 0 & 0 \\
    0 & 1.39 & 0 \\
    0 & 0 & 1
    \end{bmatrix}
    $$

    \item[Q-A5.]
    \ \\
    The RMS error was calculated in chromaticity coordinates for all 3 images, using the canonical image as a baseline. The RMS between the canonical and solux 
    image was 0.11. The RMS between the canonical and corrected image dropped to 0.05.

    \item[Q-A6.]
    \ \\
    The angular errors for the MaxRGB algorithm for the Apple, Ball, and Block images were 9.77, 3.71, 19.9 degrees respectively. Personally, it was unclear why 
    the comparison was made with the solux unknown light and not the canonical known light source.

    \item[Q-A7.]
    \ \\
    The workflow in QA5 was repeated for the Apple, Ball, and Block images and the results are shown in \autoref{fig:Figure2}. All images were scaled to a max 
    value of 250 to avoid confusion of pixel brighness, and the RMS values for each image is shown in \autoref{tab:Table1}. The blue channel appears to be 
    overcompensated for when the MaxRGB algorithm is used irrespective of the scaling factor used to minimize confusion. This results in a higher RMS error in 
    the corrected image vs the solux light in \autoref{tab:Table1}. I didn't observe any trend between the angular errors and the RMS errors. This might be 
    because I was comparing it to a fixed macbeth solux illuminant but even when I compared it to the canonical macbeth white color estimate, I didn't observe 
    a correlation.

    \begin{table}[h!]
    \begin{center}
    \begin{tabular}{ | c | c | c | } 
        \hline
        Image Name & Canonical vs Solux & Canonical vs Corrected \\ 
        \hline \hline
        Apples & 0.08 & 0.03 \\ 
        Balls  & 0.12 & 0.10 \\ 
        Blocks & 0.08 & 0.09 \\
        \hline
    \end{tabular}
    \caption{RMS error estimates for the Canonical images vs. the solux (unknown light) and MaxRGB corrected illuminant image.}
    \label{tab:Table1}
    \end{center}
    \end{table}

    \begin{figure}[H]
        \centering
        \includegraphics[scale=0.5]{output/f2_maxRGB_img_results.png}
        \caption{Comparison of Apple, Ball, and Block images arranged on each row. The bluish effect is removed from the corrected images in the middle panel. 
        Note the overly bright blue block in the 3rd row 2nd column.}
        \label{fig:Figure2}
    \end{figure}

    \FloatBarrier 

    \item[Q-A8.]
    \ \\
    The same workflow for the QA7 was repeated for the same images but with the gray-world algorithm. The angular errors for the gray-world method for the Apple, Ball, 
    and Block images were 19.7, 1.40, 23.4 degrees respectively. The RMS error values for each image is shown in \autoref{tab:Table2}. The gray-world method gives 
    lower RMS errors than the MaxRGB algorithm on this dataset and the bright blue channel amplitudes are better compensated for (\autoref{fig:Figure3}).

    \begin{table}[h!]
    \begin{center}
    \begin{tabular}{ | c | c | c | } 
        \hline
        Image Name & Canonical vs Solux & Canonical vs Corrected \\ 
        \hline \hline
        Apples & 0.08 & 0.02 \\ 
        Balls  & 0.12 & 0.09 \\ 
        Blocks & 0.08 & 0.03 \\
        \hline
    \end{tabular}
    \caption{RMS error estimates for the Canonical images vs. the solux (unknown light) and gray-world corrected illuminant image.}
    \label{tab:Table2}
    \end{center}
    \end{table}

    \begin{figure}[H]
        \centering
        \includegraphics[scale=0.5]{output/f3_gray_world_img_results.png}
        \caption{Comparison of Apple, Ball, and Block images arranged on each row. The bluish effect is removed from the corrected images in the middle panel. 
        Improved correction of bright blue amplitudes in the Ball and Block images.}
        \label{fig:Figure3}
    \end{figure}

    \FloatBarrier 

    \item[Q-A9.]
    \ \\
    I tried to derive an optimal formula to no avail, it was never as performant as the prior RMS solutions, so I just settled for the LSQR solution. Based on my
    derivation alone, it was not guaranteed to be more performant (I was too tired to try and resolve it).
    \begin{align*}
    [d_R,d_G,d_B] &= [R_U(R_C^{-1}), G_U(G_C^{-1}), B_U(B_C^{-1})] \\
    D &= 
    \begin{bmatrix}
    d_R & 0 & 0 \\
    0 & d_G & 0 \\
    0 & 0 & d_B
    \end{bmatrix}
    \end{align*}
    where \\
    - R,G,B - represent pixels across the red, green, and blue channels \\
    - U represents pixels from the unknown light \\
    - C represents pixels from the canonical light \\
    The RMS comparison between my \textbf{custom formula} and the LSQR diagonal matrix solution is shown in \autoref{tab:Table3}.

    \begin{table}[h!]
    \begin{center}
    \begin{tabular}{ | c | c | c | } 
        \hline
        Image Name & Custom Formula & LSQR Solution \\ 
        \hline \hline
        Macbeth & 0.13 & 0.05 \\ 
        Apples & 0.11 & 0.02 \\ 
        Ball & 0.13 & 0.09 \\ 
        Block & 0.18 & 0.03 \\ 
        \hline
    \end{tabular}
    \caption{RMS error estimates from using two methods to estimate the diagonal matrix. The LSQR solution is more performant than the custom formula.}
    \label{tab:Table3}
    \end{center}
    \end{table}


    \item[Q-A10.]
    \ \\
    I was encountering errors using the \emph{fminunc()} in Matlab. Irrespective of the starting point, I encountered the error below
    \begin{lstlisting}[language=Matlab]
    Initial point is a local minimum.
    Optimization completed because the size of the gradient at the 
    initial point is less than the value of the optimality tolerance.
    \end{lstlisting}
    I was able to get \emph{fminsearch()} to work. I used two starting points, namely the oracle illuminant from QA5 and the lsqr solution from Q9. The optimization 
    algorithm improved the results by a few points but the precision is below the number of significant numbers required for this report, hence the values look very similar. \\
    The RMS comparison when I used the oracle illuminant is shown in \autoref{tab:Table4}, while the RMS comparison for the lsqr diagonal matrix startpoint is shown 
    \autoref{tab:Table5}. In the report, both tables look exactly the same but the start point affects the optimal solution and the LSQR start point solution was more 
    precise than the oracle illuminant start point. I could barely improve on the results from B9.

    \begin{table}[h!]
    \begin{center}
    \begin{tabular}{ | c | c | c | } 
        \hline
        Image Name & Oracle Illuminant & fminsearch \\ 
        \hline \hline
        Macbeth & 0.05 & 0.04 \\ 
        Apples & 0.02 & 0.02 \\ 
        Ball & 0.09 & 0.09 \\ 
        Block & 0.03 & 0.03 \\ 
        \hline
    \end{tabular}
    \caption{RMS error estimates for before and after applying fminsearch to optimize the diagonal matrix. Oracle lluminant matrix is used as start point.}
    \label{tab:Table4}
    \end{center}
    \end{table}

    \begin{table}[h!]
    \begin{center}
    \begin{tabular}{ | c | c | c | } 
        \hline
        Image Name & LSQR & fminsearch \\ 
        \hline \hline
        Macbeth & 0.05 & 0.04 \\  
        Apples & 0.02 & 0.02 \\ 
        Ball & 0.09 & 0.09 \\ 
        Block & 0.03 & 0. \\ 
        \hline
    \end{tabular}
    \caption{RMS error estimates for before and after applying fminsearch to optimize the diagonal matrix. LSQR diagonal matrix is used as start point.}
    \label{tab:Table5}
    \end{center}
    \end{table}
    

\end{enumerate}

\end{document}