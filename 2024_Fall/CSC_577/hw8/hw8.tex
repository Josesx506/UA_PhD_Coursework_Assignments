%
% Latex comments start with the percent symbol.
%
% This file should create a pdf on a mac or Linux command line by running:
%     pdflatex hw8.tex
% I usually add a few options
%     pdflatex -halt-on-error -interaction=nonstopmode -file-line-error hw8.tex
% 
% If you are new to Latex, you might not know that you may need to run the above
% twice for the compiler to sort out its references. (There are ways to finesse
% this). 
%

\documentclass[12pt]{report}

% Whether or not you need all these packages, or even some more will vary. These
% are some common ones, but not all are needed for this document. There is no
% real harm loading your favorites out of habit. 


\usepackage{algorithm,algorithmic,alltt,amsmath,amssymb,bm,
    cancel,color,fullpage,graphicx,listings,mathrsfs,
    multirow,setspace,subcaption,upgreek,xcolor}
\usepackage[numbered,framed]{matlab-prettifier}
\usepackage[colorlinks]{hyperref}
\usepackage[nameinlink,noabbrev]{cleveref}
\usepackage[verbose]{placeins}
\usepackage{caption}
\usepackage[skip=0.1ex, belowskip=1ex,
            labelformat=brace,
            singlelinecheck=off]{subcaption}
\usepackage{float, wrapfig, multicol}

\doublespacing

%%%%%%%%%%%%%%%%%%%%%%%%%%%% Operators %%%%%%%%%%%%%%%%%%%%%%%%%%%%%%%
% Your personal shortcuts. You do not need to use any. argmax and argmin
\DeclareMathOperator*{\argmax}{arg\,max}
\DeclareMathOperator*{\argmin}{arg\,min}

%% Distributions
\newcommand{\N}{\mathcal{N}}
\newcommand{\U}{\mathcal{U}}
\newcommand{\Poi}{{\text Poisson}}
\newcommand{\Exp}{{\text Exp}}
\newcommand{\G}{\mathcal{G}}
\newcommand{\Ber}{{\text Bern}}
\newcommand{\Lap}{{\text Laplace}}
\newcommand{\btheta}{\boldsymbol{\theta}}
\newcommand{\bSigma}{\boldsymbol{\Sigma}}
% \usepackage[left=1cm,right=2.5cm,top=2cm,bottom=1.5cm]{geometry}
% Code blocks formatting

\definecolor{MyDarkGreen}{rgb}{0.0,0.4,0.0}

% For faster processing, load Matlab syntax for listings
\lstloadlanguages{Matlab}%
\lstset{language=Matlab,        % Use MATLAB
        % frame=single,   % Single frame around code
        basicstyle=\small\ttfamily,     % Use small true type font
        keywordstyle=[1]\color{blue}\bfseries,  % MATLAB functions bold and blue
        keywordstyle=[2]\color{purple}, % MATLAB function arguments purple
        keywordstyle=[3]\color{blue}\underbar,  % User functions underlined and blue
        identifierstyle=,       % Nothing special about identifiers
        % Comments small dark green courier
        commentstyle=\usefont{T1}{pcr}{m}{sl}\color{MyDarkGreen}\small,
        stringstyle=\color{purple},     % Strings are purple
        showstringspaces=false,         % Don't put marks in string spaces
        tabsize=2,      % 2 spaces per tab
        %%% Put standard MATLAB functions not included in the default language here
        morekeywords={xlim,ylim,var,alpha,factorial,poissrnd,normpdf,normcdf},
        %%% Put MATLAB function parameters here
        morekeywords=[2]{on, off, interp},
        %%% Put user defined functions here
        morekeywords=[3]{hw1,hw2,},
        gobble=4,
        morecomment=[l][\color{blue}]{...},     % Line continuation (...) like blue comment
        numbers=left,   % Line numbers on left
        firstnumber=1,          % Line numbers start with line 1
        numberstyle=\tiny\color{blue},  % Line numbers are blue
        stepnumber=5    % Line numbers go in steps of 5
}

%% Probability
\newcommand{\E}[1]{\mathbb{E}[#1]}
\newcommand{\Cov}[2]{\mathbb{C}\mathrm{ov}(#1,#2)}

%% Bold font for vectors from Ernesto, but I do not know how the first one
%  works, but it seems necessary for the second?
\def\*#1{\mathbf{#1}}
\newcommand*{\V}[1]{\mathbf{#1}}

%%%%%%%%%%%%%%%%%%%%%%%%%%%%%%%%%%%%%%%%%%%%%%%%%%%%%%%%%%%%%%%%%%%%%%

\begin{document}


\centerline{\it CS 577}
\centerline{\it HW \#8 Submission}
\centerline{\it Name: Joses Omojola}

Questions from Part A-C1 were completed in matlab and saved in the \emph{hw8.m} program. The program creates an \emph{output} folder to save images, so that the 
root directory is not always cluttered, and it can be run using the \textit{hw8()} command. Most of the results are exported as graphics, and only question B6 is 
printed to terminal. All numbers are rounded to \emph{3 s.f} for reporting purposes.

\begin{enumerate}

    \item[Part A.]
    \ \\
    Completed the three class examples on correlation and convolution. No discussion because we were instructed not to write it up. The results are shown in 
    \autoref{fig:Figure1}.

    \begin{figure}[H]
        \centering
        \includegraphics[scale=0.5]{output/f0_corr_vs_conv.png}
        \caption{Difference between correlation and convolution based on class examples in lecture 13.}
        \label{fig:Figure1}
    \end{figure}
    
    \item[Part B.]
    \item[B1.] Two finite difference kernels $G_x$ and $G_y$ were convolved with the input image, and the gradient was obtained by squaring the sum of convolutions 
    along the row and column dimensions. The output was scaled between 0-255, and the resulting image is shown in \autoref{fig:Figure2}.

    \noindent
    \begin{minipage}{0.3\textwidth}
        \begin{flalign*}
        &G_x = \begin{bmatrix}
            -1 & 0 & 1 \\
            -2 & 0 & 2 \\
            -1 & 0 & 1
            \end{bmatrix}
        \\
        &G_y = \begin{bmatrix}
            1 & 2 & 1 \\
            0 & 0 & 0 \\
            -1 & -2 & -1
            \end{bmatrix}
        \end{flalign*}
    \end{minipage}
    \begin{minipage}{0.7\columnwidth}
        \begin{figure}[H]
            \centering
            \includegraphics[scale=0.5]{output/f1_gradient_climber.png}
            \caption{Magnitude of the gradient of the input climber image. Edges in the image have brighter colors.}
            \label{fig:Figure2}
        \end{figure}
    \end{minipage}


    \item[B2.] Using a threshold of \textbf{35}, the edges in the image are shown in \autoref{fig:Figure3}. Values above the threshold are shown as white and smaller 
    values are shown as black.
    \begin{figure}[H]
        \centering
        \includegraphics[scale=0.5]{output/f2_edge_detection.png}
        \caption{Edge detection obtained by thresholding the magnitude of the gradient for the input climber image.}
        \label{fig:Figure3}
    \end{figure}

    \item[B3.] A gaussian filter with a standard deviation of 2 was convolved with the input image, to smooth out noisy pixels. The input gaussian mask and the resulting 
    figure are shown in \autoref{fig:Figure4}. This assumes that the noise removal process is sensitve to outliers.
    \begin{figure}[H]
        \centering
        \includegraphics[scale=0.8]{output/f3_gaussian_smoothing.png}
        \caption{Effect of gaussian smoothing on input image. Gaussian mask with $(\sigma=2)$ is shown on left, and the smoothed image with less noisy pixels is shown on the 
        right.}
        \label{fig:Figure4}
    \end{figure}

    \item[B4.] The gradient magnitude of the blurred (denoised) image was computed and a threshold of \textbf{70} was used for edge detection. The resulting image is shown 
    in \autoref{fig:Figure5}. The rolling hills in the background of the image were grayed out during the denoising operation, and are not clearly detected as edges in the 
    resulting figure. Using the same threshold as question \emph{B2} resulted in fat edges from the denoising operation, hence the higher threshold.
    \begin{figure}[H]
        \centering
        \includegraphics[scale=0.5]{output/f4_gaussian_smoothing_edge_detect.png}
        \caption{Edge detection on denoised image. A gaussian filter $(\sigma=2)$ was applied prior to estimating the gradient magnitude, and a threshold of 70 was applied 
        for edge detection.}
        \label{fig:Figure5}
    \end{figure}

    \item[B5.] To highlight the associativity of the convolution operator, the input image was filtered twice. The first filter combines the blurring and gradient filters, 
    and convolves the input image with this merged filter. The second filter does the blurring and gradient filtering steps as 2 separate operations. The gaussian filter has 
    $(\sigma=2)$, and an edge threshold of \emph{20} is used for both filters. The resulting images are shown in \autoref{fig:Figure6}. Because convolution is associative, both 
    images look very similar irrespective of whether the filters are combined or applied separately.
    \begin{figure}[H]
        \centering
        \includegraphics[scale=0.7]{output/f5_filter_comparison.png}
        \caption{Gradient magnitude on denoised image with gaussian filter $(\sigma=4)$ and edge threshold of 20. The blurring and edge detection filters are merged on the left 
        image, and applied separately for the right image. Both operations provide similar results.}
        \label{fig:Figure6}
    \end{figure}

    \item[B6.] A 2D Gaussian function with standard deviation \(\sigma\) can be expressed as:
    \[
    f(x, y) = \frac{1}{2 \pi \sigma^2} e^{-\frac{x^2 + y^2}{2 \sigma^2}}
    \]

    To show separability, we can rewrite \(f(x, y)\) as a product of two 1D Gaussian functions.
    \[
    f(x, y) = \frac{1}{2 \pi \sigma^2} e^{-\frac{x^2}{2 \sigma^2}} e^{-\frac{y^2}{2 \sigma^2}}
    \]
    This can be separated into:
    \[
    f(x, y) = g(x) \cdot h(y)
    \]
    where:
    \[
    g(x) = \frac{1}{\sqrt{2 \pi} \sigma} e^{-\frac{x^2}{2 \sigma^2}}, \quad h(y) = \frac{1}{\sqrt{2 \pi} \sigma} e^{-\frac{y^2}{2 \sigma^2}}
    \]
    \( g(x) \) and \( h(y) \) are 1D Gaussian functions with standard deviation \(\sigma\). This allows us to convolve the input image rows with \( g(x) \), and the resulting 
    matrix with \( h(y) \) along each column, as two separate convolution operations.  

    A direct 2D convolution has $O(n \times m \times k^2)$ complexity, where $n \times m$ is the image size and $k \times k$ is the kernel size. Using separate filters, the 
    complexity is reduced to $O(n \times m \times k + n \times m \times k)$, since each 1D convolution (horizontal and vertical) has only $O(n \times m \times k)$ complexity. 
    This means the 1D convolution is faster, and we can obtain comparable results as shown in \autoref{fig:Figure7}. When using a gaussian filter with $\sigma=20$ in matlab, 
    the 1D convolution is faster by \textit{\textbf{0.03}} seconds.

    \begin{figure}[H]
        \centering
        \includegraphics[scale=0.7]{output/f6_convolution_comparison.png}
        \caption{Comparison of 1D vs. 2D convolution with a gaussian filter $(\sigma=4)$. Both operations output similar results however, the 1D convolution has a faster runtime.}
        \label{fig:Figure7}
    \end{figure}

    \item[Part C1.]
    \ \\
    The gradient based edge detector was implemented in matlab. By providing an input $\sigma$ value, the input image is first smoothed with a gaussian filter to surpress 
    noisy pixels, and then differentiated to highlights edges. The degree of smoothing affects the recovered edges. The gradient magnitude and direction are estimated, and 
    the gradient direction is used to create a neighboring pixels matrix, showing diagonal angles from $0-360$ at increments of $45^o$. Using non-max-suppression (nms), the 
    neighboring pixel matrix is used to identify the maximum gradient magnitude values along the gradient which helps to highlight edge points. Because the maximum value is dynamic 
    depending on the image, I set it to a percentage of the \emph{nms} matrix. This percentage can be altered in the function input arguments to increase or decrease the 
    edge detection threshold. Because edge points form along curves, we can construct the tangent to the edge curve using the neighbor matrix and try to predict where the 
    next edge point should be in a process known as edge-linking. To implement this, I set a weak edge percentage threshold, and a strong edge threshold. Pixels in the nms matrix 
    that are $ \textit{pixel} \ge \textit{lower threshold} < \textit{upper threshold}$, are assigned as weak edges, while all the pixels stronger than the \textit{upper threshold} 
    have strong edges. The program loops through each pixel and checks whether it is a strong edge or not. It then looks at the eight neighboring pixels to identify any weak 
    edges, and using the weak edge matrix to link nearby edge points that are not very clear.  
    
    This creates a two-fold workflow, where non-max-suppression reduces the width of blurred edges to identify the max edge, and the edges are linked to create continuous edges 
    in images. Tests were run for 3 different scenarios and the results are shown in  \autoref{fig:Figure8}.

    \begin{figure}[H]
        \centering
        \includegraphics[scale=0.7]{output/f7_gradient_based_edge_detec.png}
        \caption{Gradient-based edge detection with linking applied. (a) Fine scale smoothing with high threshold. (b) Coarse scale smoothing with high threshold. (c) Coarse scale 
        smoothing with low threshold.}
        \label{fig:Figure8}
    \end{figure}

    When fine-scale smoothing is applied (smaller $\sigma$), the resolved edges are thinner and small edges can be easily resolved. However, depending on the feature of interest, it 
    can create noisy edges as seen along the rock face in \emph{Figure 8a}. Coarse filters with high-thresholds can be used to suppress the small-scale edges along the rock face, and 
    allows us to focus on larger features like the background mountains, however, smaller features like the climbers outline are poorly resolved (\emph{Figure 8b}). Lowering the 
    threshold when using a coarse filter can improve resolution of smaller edge features like the background mountains / rock face while still suppressing extremely small edge features 
    that can make the image noisy (\emph{Figure 8c}). In the end, the choice of the filter scale and threshold depends on the structure scale that the user wants to highlight. Tweaking 
    these values affects which edges come into focus, and which edges are blurred out.  
    
    Overall, the gradient-based edge linking filter gives thinner edges that are better connected, than edge-detectors that do not implement linking \autoref{fig:Figure9}. Edges 
    are better resolved when linked, and they tend to extend across longer pixel distances.

    \begin{figure}[H]
        \centering
        \includegraphics[scale=0.7]{output/f8_linked_vs_unlinked.png}
        \caption{Effect of linking on edge-detection. Both images have the same degree of gaussian smoothing, however, the left image has linked edges and provides higher resolution 
        results than the unlinked edge detector algorithm (right), with lower edge artifacts at the image boundaries.}
        \label{fig:Figure9}
    \end{figure}


    

\end{enumerate}

\end{document}