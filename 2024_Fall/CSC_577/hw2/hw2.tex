%
% Latex comments start with the percent symbol.
%
% This file should create a pdf on a mac or Linux command line by running:
%     pdflatex hw1.tex
% I usually add a few options
%     pdflatex -halt-on-error -interaction=nonstopmode -file-line-error hw1.tex
% 
% If you are new to Latex, you might not know that you may need to run the above
% twice for the compiler to sort out its references. (There are ways to finesse
% this). 
%

\documentclass[12pt]{report}

% Whether or not you need all these packages, or even some more will vary. These
% are some common ones, but not all are needed for this document. There is no
% real harm loading your favorites out of habit. 


\usepackage{algorithm,algorithmic,alltt,amsmath,amssymb,bm,
    cancel,color,fullpage,graphicx,listings,mathrsfs,
    multirow,setspace,subcaption,upgreek,xcolor}
\usepackage[numbered,framed]{matlab-prettifier}
\usepackage[colorlinks]{hyperref}
\usepackage[nameinlink,noabbrev]{cleveref}
\usepackage[verbose]{placeins}

\doublespacing

%%%%%%%%%%%%%%%%%%%%%%%%%%%% Operators %%%%%%%%%%%%%%%%%%%%%%%%%%%%%%%
% Your personal shortcuts. You do not need to use any. argmax and argmin
\DeclareMathOperator*{\argmax}{arg\,max}
\DeclareMathOperator*{\argmin}{arg\,min}

%% Distributions
\newcommand{\N}{\mathcal{N}}
\newcommand{\U}{\mathcal{U}}
\newcommand{\Poi}{{\text Poisson}}
\newcommand{\Exp}{{\text Exp}}
\newcommand{\G}{\mathcal{G}}
\newcommand{\Ber}{{\text Bern}}
\newcommand{\Lap}{{\text Laplace}}
\newcommand{\btheta}{\boldsymbol{\theta}}
\newcommand{\bSigma}{\boldsymbol{\Sigma}}
% \usepackage[left=1cm,right=2.5cm,top=2cm,bottom=1.5cm]{geometry}
% Code blocks formatting

\definecolor{MyDarkGreen}{rgb}{0.0,0.4,0.0}

% For faster processing, load Matlab syntax for listings
\lstloadlanguages{Matlab}%
\lstset{language=Matlab,        % Use MATLAB
        % frame=single,   % Single frame around code
        basicstyle=\small\ttfamily,     % Use small true type font
        keywordstyle=[1]\color{blue}\bfseries,  % MATLAB functions bold and blue
        keywordstyle=[2]\color{purple}, % MATLAB function arguments purple
        keywordstyle=[3]\color{blue}\underbar,  % User functions underlined and blue
        identifierstyle=,       % Nothing special about identifiers
        % Comments small dark green courier
        commentstyle=\usefont{T1}{pcr}{m}{sl}\color{MyDarkGreen}\small,
        stringstyle=\color{purple},     % Strings are purple
        showstringspaces=false,         % Don't put marks in string spaces
        tabsize=2,      % 2 spaces per tab
        %%% Put standard MATLAB functions not included in the default language here
        morekeywords={xlim,ylim,var,alpha,factorial,poissrnd,normpdf,normcdf},
        %%% Put MATLAB function parameters here
        morekeywords=[2]{on, off, interp},
        %%% Put user defined functions here
        morekeywords=[3]{hw1,hw2,},
        gobble=4,
        morecomment=[l][\color{blue}]{...},     % Line continuation (...) like blue comment
        numbers=left,   % Line numbers on left
        firstnumber=1,          % Line numbers start with line 1
        numberstyle=\tiny\color{blue},  % Line numbers are blue
        stepnumber=5    % Line numbers go in steps of 5
}

%% Probability
\newcommand{\E}[1]{\mathbb{E}[#1]}
\newcommand{\Cov}[2]{\mathbb{C}\mathrm{ov}(#1,#2)}

%% Bold font for vectors from Ernesto, but I do not know how the first one
%  works, but it seems necessary for the second?
\def\*#1{\mathbf{#1}}
\newcommand*{\V}[1]{\mathbf{#1}}

%%%%%%%%%%%%%%%%%%%%%%%%%%%%%%%%%%%%%%%%%%%%%%%%%%%%%%%%%%%%%%%%%%%%%%

\begin{document}


\centerline{\it CS 577}
\centerline{\it HW \#2 Submission}
\centerline{\it Name: Joses Omojola}

Eight out of 9 questions were answered in the \emph{hw2.m} file. The file creates an \emph{output} folder to save images, 
so that the root directory is not always cluttered. Figures are saved and automatically closed when the script is run to
minimize the number of open windows. The script can be run by specifying all the names of the 3 files as input arguments 
to the function $ n\_qsts = hw2('rgb\_sensors.txt','light\_spectra.txt','responses.txt');$. Programming was completed with 
Matlab R2024a, and it took me roughly 20 hours to complete the assignment (Coding + Exposition).

Text files were read into matlab using the \emph{importdata()} function, nested functions were used to reduce repeated 
code, and subplots were used where necessary to minimize output files. Titles were included to differentiate the purpose 
of each subplot and question numbers are denoted with a "Q" prefix in the text below. The bonus question was omitted due 
to time constraints.

\begin{enumerate}

    \item[Q1.]

    The random seed was set to "477" for reproducibility and the input sensors are from the \emph{rgb\_sensors.txt} file is shown 
    below in \autoref{fig:Figure1}. The blue channel is most sensitive at short wavelengths, while the red channel is 
    sensitive at longer wavelengths.

    The simulated light spectra was scaled by \textbf{9.4e-4} to adjust the input range between 0-255. The RGB response was obtained 
    by multiplying the spectra with the sensor sensitivity. The array was reshaped to match the assignment instructions and the 
    results are shown in \autoref{fig:Figure1}. The sensitivity spectrum is optimized to improve indoor lighting hence the bias 
    towards blue.

    \begin{figure}[ht!]
        \centering
        \includegraphics[scale=0.5]{output/f1_rgb.png} \hfill
        \includegraphics[scale=0.95]{output/f2_rgb400by400.png}
        \caption{Q1 output images (a) Input sensor sensitivity spectra. 
        (b) RGB image from simulated light spectra.}
        \label{fig:Figure1}
    \end{figure}

    \FloatBarrier 

    \item[Q2.]

    The least-squares method was used to invert for the sensor sensitivities with the \emph{linsolve()} function in matlab.
    The inverted results are compared to the original sensitivities in \autoref{fig:Figure2}. A near perfect match was obtained 
    and a direct overlay didn't show any differences, hence the split across the multiple subplots.

    \begin{figure}[ht!]
        \centering
        \includegraphics[scale=0.6]{output/f3_inverted_spectra.png}
        \caption{Ground truth vs. inverted sensor sensitivities.}
        \label{fig:Figure2}
    \end{figure}

    \FloatBarrier

    The summary of the obtained errors are shown in \autoref{tab:Table1}. The blue channel spectrum has the highest error, while 
    the errors are generally lower across other channels including the total rgb response error. From the minor errors encountered, 
    the estimated sensors can precisely reconstruct the RGB.

    \begin{table}[h!]
    \begin{center}
    \begin{tabular}{ c | c | c | c | c} 
        \hline
        & Red & Green & Blue & RGB \\ 
        \hline \hline
        RMS Errors & 3.4954e-12 & 4.2429e-12 & 1.1388e-11 & 4.9092e-14 \\ 
        \hline
    \end{tabular}
    \caption{RMS errors from inverted sensor spectra.}
    \label{tab:Table1}
    \end{center}
    \end{table}

    \item[Q3.]

    Additional tests were run by adding random noise with a standard deviation of 10 to the estimated rgb response while inverting 
    for the sensor sensitivity spectra. Comparison between unclipped and clipped rgb values before inverting for sensitivity is 
    shown in \autoref{fig:Figure3}.

    \begin{figure}[ht!]
        \centering
        \includegraphics[scale=0.6]{output/f4_inverted_spectra_noise10.png}
        \caption{Inverted sensor sensitivities after adding random noise.}
        \label{fig:Figure3}
    \end{figure}

    \FloatBarrier

    The summary RMS errors are shown in \autoref{tab:Table2}. At low noise standard deviations, the influence of clipping is 
    minimal on the sensors, however; the reconstructed RGB from the clipped input is more accurate.

    \begin{table}[h!]
    \begin{center}
    \begin{tabular}{ c | c | c | c | c | c} 
        \hline
        & Red & Green & Blue & Sensor & RGB \\ 
        \hline \hline
        Unclipped & 1041.8 & 999.77 & 877.13 & 975.4 & 9.6859 \\ 
        Clipped & 1041.8 & 999.77 & 882.61 & 977.05 & 9.6777  \\
        \hline
    \end{tabular}
    \caption{RMS errors from inverted sensor spectra after adding random noise.}
    \label{tab:Table2}
    \end{center}
    \end{table}

    \item[Q4.]

    The noise levels were consistently increased to identify thresholds at which the rgb cannot be accurately recovered and 
    extend the tests about clipped vs. unclipped RGB levels. Values of the noise std. were increased according to the intervals 
    specified in \autoref{tab:Table3}.

    \begin{table}[h!]
    \begin{center}
    \begin{tabular}{| c | c | c | c | c |} 
        \hline
        Noise Std & Unclip. RGB RMS & Clip RGB RMS & Unclip. Sensor RMS & Clip Sensor RMS\\ 
        \hline \hline
        0  &  4.9e-14 & 4.9e-14 & 7.3e-12 & 7.3e-12  \\ 
        10 &  9.6859 &   9.6777 &  975.4 &  977.05  \\ 
        20 &  19.372 &   19.252 & 1950.8 & 1944  \\ 
        30 &  29.058 &   28.131 & 2926.2 & 2836.7  \\ 
        40 &  38.744 &    36.07 & 3901.6 & 3623.3  \\ 
        50 &   48.43 &   43.273 &   4877 & 4338.7  \\ 
        60 &  58.116 &   49.932 & 5852.4 & 5020.7  \\ 
        70 &  67.802 &    56.21 & 6827.8 & 5653.8  \\ 
        80 &  77.487 &    61.94 & 7803.2 & 6229.1  \\ 
        90 &  87.173 &   67.223 & 8778.6 & 6767.8  \\ 
        100 & 96.859 &   71.913 &   9754 & 7265.8  \\ 
        120 & 116.23 &   79.811 &  11705 & 8092.3  \\ 
        140 &  135.6 &   85.853 &  13656 & 8773.4  \\ 
        160 & 154.97 &   90.611 &  15606 & 9347.4  \\ 
        180 & 174.35 &    94.45 &  17557 & 9828.4  \\ 
        200 & 193.72 &   97.586 &  19508 & 10232  \\ 
        250 & 242.15 &   103.12 &  24385 & 10963  \\ 
        300 & 290.58 &   106.73 &  29262 & 11398  \\ 
        350 & 339.01 &   109.41 &  34139 & 11697  \\ 
        400 & 387.44 &   111.37 &  39016 & 11918  \\ 
        \hline
    \end{tabular}
    \caption{RMS errors increasing noise levels.}
    \label{tab:Table3}
    \end{center}
    \end{table}

    At high noise levels, like standard deviations of 50 and 100, the recovered sensor sensitivity is very noisy and hardly 
    accurate \autoref{fig:Figure4}. Below a noise std. of 30, the clipped and unclipped rms errors are comparable, 
    however the clipped RGB level errors begin to plateau as the noise std. approaches 200, while the unclipped results 
    continue to linearly increase \autoref{fig:Figure5}. 

    \begin{figure}[ht!]
        \centering
        \includegraphics[scale=0.6]{output/f5_inverted_spectra_noise50.png} \vfill \null 
        \includegraphics[scale=0.6]{output/f6_inverted_spectra_noise100.png}
        \caption{Recovered sensor sensitivities at varying noise levels.}
        \label{fig:Figure4}
    \end{figure}

    \begin{figure}[ht!]
        \centering
        \includegraphics[scale=0.6]{output/f7_rms_err_vs_noise.png}
        \caption{Comparison of clipped and unclipped RMS errors from inverted sensitivity variation with simulated noise.}
        \label{fig:Figure5}
    \end{figure}

    \FloatBarrier

    \item[Q5.] 

    Gamma correction depends on the relationship $I_{perceived} = I_{raw}^{\gamma}$ between the input signal and the 
    displayed intensity. 

    \begin{itemize}
        \item[-] $I_{perceived}$ is the perceived intensity (normalized between 0 and 1).
        \item[-] $I_{raw}$ is the actual raw intensity value (also normalized between 0 and 1).
        \item[-] $\gamma$ is the gamma value.
    \end{itemize}

    The perceived intensity of 80 is normalized by 255 to get $I_{perceived} = 0.3137$. The gamma value can be derived 
    by taking the log of both sides 
    $$
    \gamma = \frac {log(I_{perceived})} {{log(I_{raw})}} = \frac {log(0.3137)} {{log(0.5)}} = 1.67
    $$
    The gamma value of 1.67 indicates that the perceived intensity is disproportionately lower than the raw intensity.
    By applying a  correction with this value, the display's response can be adjusted to make the perceived intensity 
    more linear with the raw intensity.

    \item[Q6.] 

    The camera sensitivity in Q6 was inverted using the files provided for HW2. The recovered sensitivity is noisy and 
    notably different from the original sensitivities \autoref{fig:Figure6}. The overall sensor RMS error obtained was 
    \emph{1013791.700635} while a high RGB RMS value of \emph{11.979092} was obtained.


    \begin{figure}[ht!]
        \centering
        \includegraphics[scale=0.6]{output/f8_inverted_light_spectra.png}
        \caption{Inverted sensor sensitivity from light sensor dataset.}
        \label{fig:Figure6}
    \end{figure}

    







\end{enumerate}

\end{document}