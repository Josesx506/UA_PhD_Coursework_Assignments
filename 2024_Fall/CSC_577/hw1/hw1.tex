%
% Latex comments start with the percent symbol.
%
% This file should create a pdf on a mac or Linux command line by running:
%     pdflatex hw00-example-writeup.pdf
% I usually add a few options
%     pdflatex -halt-on-error -interaction=nonstopmode -file-line-error hw00-example-writeup.pdf
%
% I do not know much about Latex on Windows machines, but see:
%       https://www.latex-project.org/get/
%
% If you are new to Latex, you might not know that you may need to run the above
% twice for the compiler to sort out its references. (There are ways to finesse
% this). 
%

\documentclass[12pt]{report}

% Whether or not you need all these packages, or even some more will vary. These
% are some common ones, but not all are needed for this document. There is no
% real harm loading your favorites out of habit. 

\usepackage{fullpage}
\usepackage{amsmath,amssymb,bm,upgreek,mathrsfs}
\usepackage{algorithmic,algorithm}
\usepackage{graphicx,subcaption}
\usepackage{setspace}
\usepackage{color}
\usepackage{multirow}
\usepackage{alltt}
\usepackage{cancel}
\usepackage{listings}

\doublespacing

%%%%%%%%%%%%%%%%%%%%%%%%%%%% Operators %%%%%%%%%%%%%%%%%%%%%%%%%%%%%%%
%
% Your personal shortcuts. You do not need to use any. 
%
% argmax and argmin
\DeclareMathOperator*{\argmax}{arg\,max}
\DeclareMathOperator*{\argmin}{arg\,min}

%% Distributions
\newcommand{\N}{\mathcal{N}}
\newcommand{\U}{\mathcal{U}}
\newcommand{\Poi}{{\text Poisson}}
\newcommand{\Exp}{{\text Exp}}
\newcommand{\G}{\mathcal{G}}
\newcommand{\Ber}{{\text Bern}}
\newcommand{\Lap}{{\text Laplace}}
\newcommand{\btheta}{\boldsymbol{\theta}}
\newcommand{\bSigma}{\boldsymbol{\Sigma}}

%% Probability
\newcommand{\E}[1]{\mathbb{E}[#1]}
\newcommand{\Cov}[2]{\mathbb{C}\mathrm{ov}(#1,#2)}

%% Bold font for vectors from Ernesto, but I do not know how the first one
%  works, but it seems necessary for the second?
\def\*#1{\mathbf{#1}}
\newcommand*{\V}[1]{\mathbf{#1}}

%%%%%%%%%%%%%%%%%%%%%%%%%%%%%%%%%%%%%%%%%%%%%%%%%%%%%%%%%%%%%%%%%%%%%%

\begin{document}

\centerline{\it CS 577}
\centerline{\it HW \#1 Submission}
\centerline{\it Name: Joses Omojola}


% Q17 for origin rotated center matrix
% - Describe the New Coordinate System and Expected Covariance Matrix
% In the new coordinate system, you would expect the covariance matrix to be 
% diagonal because the data is aligned with the principal components:

% The first principal component (aligned with the largest variance) will show 
% high variance, while the second component (which should capture noise) 
% will show much smaller variance.


% We expect the new X-axis to have high variance (dominant eigenvector)
% and the new Y-axis to have low variance (remaining noise). 
% Therefore, the off-diagonal elements of the covariance matrix should
% be close to zero, indicating minimal correlation between the new axes.
% the variance along the second principal component (much smaller, 
% ideally near zero if the data is truly one-dimensional


\end{document}