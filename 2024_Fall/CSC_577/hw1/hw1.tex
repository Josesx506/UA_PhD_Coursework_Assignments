%
% Latex comments start with the percent symbol.
%
% This file should create a pdf on a mac or Linux command line by running:
%     pdflatex hw1.tex
% I usually add a few options
%     pdflatex -halt-on-error -interaction=nonstopmode -file-line-error hw1.tex
% 
% If you are new to Latex, you might not know that you may need to run the above
% twice for the compiler to sort out its references. (There are ways to finesse
% this). 
%

\documentclass[12pt]{report}

% Whether or not you need all these packages, or even some more will vary. These
% are some common ones, but not all are needed for this document. There is no
% real harm loading your favorites out of habit. 


\usepackage{algorithm,algorithmic,alltt,amsmath,amssymb,bm,
            cancel,color,fullpage,graphicx,listings,mathrsfs,
            multirow,setspace,subcaption,upgreek,xcolor}
\usepackage[numbered,framed]{matlab-prettifier}
\usepackage[colorlinks]{hyperref}
\usepackage[nameinlink,noabbrev]{cleveref}

\doublespacing

%%%%%%%%%%%%%%%%%%%%%%%%%%%% Operators %%%%%%%%%%%%%%%%%%%%%%%%%%%%%%%
% Your personal shortcuts. You do not need to use any. argmax and argmin
\DeclareMathOperator*{\argmax}{arg\,max}
\DeclareMathOperator*{\argmin}{arg\,min}

%% Distributions
\newcommand{\N}{\mathcal{N}}
\newcommand{\U}{\mathcal{U}}
\newcommand{\Poi}{{\text Poisson}}
\newcommand{\Exp}{{\text Exp}}
\newcommand{\G}{\mathcal{G}}
\newcommand{\Ber}{{\text Bern}}
\newcommand{\Lap}{{\text Laplace}}
\newcommand{\btheta}{\boldsymbol{\theta}}
\newcommand{\bSigma}{\boldsymbol{\Sigma}}
% \usepackage[left=1cm,right=2.5cm,top=2cm,bottom=1.5cm]{geometry}
% Code blocks formatting

\definecolor{MyDarkGreen}{rgb}{0.0,0.4,0.0}

% For faster processing, load Matlab syntax for listings
\lstloadlanguages{Matlab}%
\lstset{language=Matlab,                        % Use MATLAB
        % frame=single,                           % Single frame around code
        basicstyle=\small\ttfamily,             % Use small true type font
        keywordstyle=[1]\color{blue}\bfseries,  % MATLAB functions bold and blue
        keywordstyle=[2]\color{purple},         % MATLAB function arguments purple
        keywordstyle=[3]\color{blue}\underbar,  % User functions underlined and blue
        identifierstyle=,                       % Nothing special about identifiers
                                                % Comments small dark green courier
        commentstyle=\usefont{T1}{pcr}{m}{sl}\color{MyDarkGreen}\small,
        stringstyle=\color{purple},             % Strings are purple
        showstringspaces=false,                 % Don't put marks in string spaces
        tabsize=2,                              % 2 spaces per tab
        %%% Put standard MATLAB functions not included in the default language here
        morekeywords={xlim,ylim,var,alpha,factorial,poissrnd,normpdf,normcdf},
        %%% Put MATLAB function parameters here
        morekeywords=[2]{on, off, interp},
        %%% Put user defined functions here
        morekeywords=[3]{hw1,hw2,},
        gobble=4,
        morecomment=[l][\color{blue}]{...},     % Line continuation (...) like blue comment
        numbers=left,                           % Line numbers on left
        firstnumber=1,                          % Line numbers start with line 1
        numberstyle=\tiny\color{blue},          % Line numbers are blue
        stepnumber=5                            % Line numbers go in steps of 5
}

%% Probability
\newcommand{\E}[1]{\mathbb{E}[#1]}
\newcommand{\Cov}[2]{\mathbb{C}\mathrm{ov}(#1,#2)}

%% Bold font for vectors from Ernesto, but I do not know how the first one
%  works, but it seems necessary for the second?
\def\*#1{\mathbf{#1}}
\newcommand*{\V}[1]{\mathbf{#1}}

%%%%%%%%%%%%%%%%%%%%%%%%%%%%%%%%%%%%%%%%%%%%%%%%%%%%%%%%%%%%%%%%%%%%%%

\begin{document}


\centerline{\it CS 577}
\centerline{\it HW \#1 Submission}
\centerline{\it Name: Joses Omojola}

All questions are answered in the \emph{hw1.m} file. The file creates an \emph{output} folder to save images, so that 
the root directory is not always cluttered. I answered all the questions for the homework. Programming was completed 
with Matlab R2024a, and it took me roughly \textbf{30 hours} to complete the assignment (Coding + Exposition).

Images were read into matlab using the \emph{imread()} function, preprocessing was implemented to address specific
questions, and the results were exported using different matlab functions. Subplots were used where necessary to 
minimize output files and titles were included to differentiate the purpose of each subplot. Question numbers are
denoted with a "Q" prefix in the text below. Questions without deliverables were done in matlab but not addressed here.

\begin{enumerate}

    \item[Q7-10.]

    The result of \emph{help hw1} documentation is shown below.
    \begin{lstlisting}[language=Matlab]
    >> help hw1
    This function reads in an image, preprocesses them as arrays, 
    and outputs several images within the active directory. 
    Additional text is printed for some internal calculations 
    that were required. 
    Most of the syntax is focused on manipulating image arrays, 
    but the final two questions are related to PCA analysis.
    Input arg is a string path to the tent figure, and the 
    function returns the total number of questions that were 
    answered.
    \end{lstlisting}

    The output of \emph{whos()} is shown below 
    \begin{lstlisting}[language=Matlab]
    3x1 struct array with fields:
    name
    size
    bytes
    class
    global
    sparse
    complex
    nesting
    persistent
          name: 'tent'
          size: [489 728 3]
         bytes: 1067976
         class: 'uint8'
        global: 0
        sparse: 0
       complex: 0
       nesting: [1x1 struct]
    persistent: 0
    \end{lstlisting}

    The array dimensions for the image can be extracted with \emph{'[num\_rows, num\_cols, 
    num\_channel] = size(tent);'}, where the \emph{tent} variable is the array of image channels.
    
    \begin{table}[h!]
    \begin{center}
    \begin{tabular}{ ||c c c|| } 
        \hline
        Channel & Min & Max \\ 
        \hline \hline
        Red & 0 & 251 \\ 
        Green & 0 & 248 \\ 
        Blue & 0 & 253 \\ 
        Overall & 0 & 253 \\ 
        \hline
    \end{tabular}
    \caption{Summary statistics of image color channels.}
    \label{tab:Table1}
    \end{center}
    \end{table}

    \begin{figure}[ht!]
        \includegraphics[scale=0.4]{output/bw_tent.jpg}
        \centering
        \caption{Grayscale image of tent.}
        \label{fig:Figure1}
    \end{figure}

    \newpage
    The min- and max ranges of each channel is displayed below in \autoref{tab:Table1}, and 
    the grayscale image of the flattened channels is shown in \autoref{fig:Figure1}. Unlike RGB
    images, grayscale images are flattened along the 3\textsuperscript{rd} axis. Bright colors
    are lightened in grayscale mode, while darker colors maintain a deeper tone. 

    \begin{figure}[ht!]
        \includegraphics[scale=0.65]{output/tent_rgb_channels.png}
        \centering
        \caption{Split grayscale images of each channel band.}
        \label{fig:Figure2}
    \end{figure}

    Individual channels can also be split to create multiple grayscale channel bands. An
    example is shown in \autoref{fig:Figure2}. The yellow tent is bright on the red and 
    green channels because both colors can be combined to create a yellow mix. The tent 
    is dark on the blue channel because the blue channel is not sensitive to signals 
    within the bright yellow wavelength. When the channels are exchanged with each other, 
    the colors of items take on a different color e.g. the originally yellow tent becomes 
    purple after the channels are swapped in \autoref{fig:Figure3}.

    \begin{figure}[ht!]
        \includegraphics[scale=0.4]{output/tent_flip_ch.png}
        \centering
        \caption{Altered channels from original tent image.}
        \label{fig:Figure3}
    \end{figure}

    When displaying images, digital representations can apply scaling to generate false colors
    across channels. e.g. Using the \emph{imagesc()} function to plot an image, scales the image 
    color by the selected colormap min and max irrespective of the matrix values. The \emph{imshow()}
    function on the other hand shows grayscaled images scaled using white and black to represent 
    light and dark patches across an input array. An example is shown below in \autoref{fig:Figure4}.

    \newpage
    \begin{figure}[ht!]
        \includegraphics[scale=0.8]{output/tent_checker_bw_for_loop.png}
        \centering
        \caption{False color representation of black and white image for white squares.}
        \label{fig:Figure4}
    \end{figure}

    \emph{imshow()} is better suited for representing black and white images because it doesn't render 
    false colors of the assigned colormap like \emph{imagesc()} does. The matrix manipultation in \emph{Q10}
    can be replicated using array masking. In the snippet below, we make 1/25\textsuperscript{th} of all
    the cells in the image matrix black.

    \begin{lstlisting}[language=Matlab]
        doub_bw_tent = double(bw_tent) / 255;
        [nRows, nCols] = size(doub_bw_tent);
        doub_bw_tent(5:5:nRows, 5:5:nCols) = 0; % array mask
    \end{lstlisting}

    \item[Q16.]

    The result of altering the arrays \autoref{fig:Figure5} highlight a similar false color issue of the 
    \emph{imagesc()} function observed in \emph{Q10}. Masking makes it easier to target a specific range
    of values in an array. This can be used to improve color correction during preprocessing of images.
    Another example of masking arrays is shown in \autoref{fig:Figure6}. Extremely bright values in the 
    grayscale image from the snow and tent pixels are easily manipulated to become dark. 

    \begin{figure}[ht!]
        \includegraphics[scale=0.8]{output/tent_checker_bw_vectorized.png}
        \centering
        \caption{False color representation of black and white image for black squares.}
        \label{fig:Figure5}
    \end{figure}

    \begin{figure}[ht!]
        \includegraphics[scale=0.4]{output/tent_bw_0.5_mask.png}
        \centering
        \caption{Masked bright colors in BW image.}
        \label{fig:Figure6}
    \end{figure}

    \item[Q11.]

    Historams are simplified method for showing the Linear sensor response of a camera. They can highlight 
    peaks within which each channel is sensitive to, and provide a 1D prior for developing models of the 
    world from digital images. Channel histograms are shown in \autoref{fig:Figure7}.
    
    \begin{figure}[ht!]
        \includegraphics[scale=0.65]{output/tent_rgb_hist.png}
        \centering
        \caption{Histogram response of rgb channels.}
        \label{fig:Figure7}
    \end{figure}

    \newpage
    \item[Q12.]

    Plots in matlab can be easily made using the \emph{plot()} function \autoref{fig:Figure8}.
    
    \begin{figure}[ht!]
        \includegraphics[scale=0.5]{output/trig_waves.png}
        \centering
        \caption{Plot of sine and cosine wave.}
        \label{fig:Figure8}
    \end{figure}

    \item[Q13-15.] Linear Algebra
    \emph{inv()} inverts the values for x,y,z to be x = \textbf{1.9375}, y = \textbf{0.2500}, z = \textbf{2.1875}.
    The same values are gotten with \emph{linsolve()}, and proof of correctness was determined by substituting
    the outputs into one of the equations.  Subtraction of the results from both methods is zero, indicating that they
    are mathematically equivalent for solving linear equations. From documentation, \emph{linsolve()} is better 
    suited at solving larger linear equations effectively. Questions 14 - 15 were solved in the code but not 
    described because they have no deliverables.

    \item[Q17-18.] PCA

    The plot of the two variables in the \emph{pca.txt} fie are shown in \autoref{fig:Figure9}. 
    
    \begin{figure}[ht!]
        \includegraphics[scale=0.5]{output/pca_raw.png}
        \centering
        \caption{Cross-plot of PCA data scatter.}
        \label{fig:Figure9}
    \end{figure}

    The covariance matrix of the data is 
    $$
    \begin{bmatrix}
    0.0211 & 0.0356 \\
    0.0356 & 0.0608
    \end{bmatrix}
    $$

    \newpage
    After rotating the data to the new coordinate system, the covariance matrix is expected to be 
    diagonal because the data is aligned with the principal components. The first principal component 
    (aligned with the largest variance) will show high variance, while the second component (which 
    should capture noise) will show much smaller variance. Therefore, the off-diagonal elements of the 
    covariance matrix should be close to zero, indicating minimal correlation between the new axes,if 
    the data is truly one-dimensional.

    After mean centering the data and transforming the data by the rotation matrix, we get,
    $$
    M = 
    \begin{bmatrix}
    -0.8624 &  0.5063 \\
    0.5063  &  0.8624
    \end{bmatrix}
    $$
    Proof of orthogonality can be determined by multiplying the matrix with its inverse $I = M' \times M$ to get an 
    identity matrix
    $$
    I = 
    \begin{bmatrix}
    1.0000  &      0 \\
    0       & 1.0000
    \end{bmatrix}
    $$
    The transformed data is shown in \autoref{fig:Figure10}. 
    
    \begin{figure}[ht!]
        \includegraphics[scale=0.5]{output/pca_mean_cent.png}
        \centering
        \caption{Transformed data into principal component dimentsions.}
        \label{fig:Figure10}
    \end{figure}

    The covariance matrix for the transformed data is 
    $$
    \begin{bmatrix}
    0.0001  &  0.0000 \\
    0.0000  &  0.0818
    \end{bmatrix}
    $$
    The sum of the variance values from before and after the transformation is the same as shown 
    in \autoref{tab:Table2}. Similar variance sums were obtained because the transformation is 
    orthogonal, which preserves the total variance of the data.

    \begin{table}[h!]
    \begin{center}
    \begin{tabular}{ c | c | c } 
        \hline
         & Before & After \\ 
        \hline \hline
        Variance Sums & 0.081909 & 0.081909 \\ 
        \hline
    \end{tabular}
    \caption{Sum of variance values from cov. matrices.}
    \label{tab:Table2}
    \end{center}
    \end{table}

\end{enumerate}

\end{document}