%
% Latex comments start with the percent symbol.
%
% This file should create a pdf on a mac or Linux command line by running:
%     pdflatex hw1.tex
% I usually add a few options
%     pdflatex -halt-on-error -interaction=nonstopmode -file-line-error hw1.tex
% 
% If you are new to Latex, you might not know that you may need to run the above
% twice for the compiler to sort out its references. (There are ways to finesse
% this). 
%

\documentclass[12pt]{report}

% Whether or not you need all these packages, or even some more will vary. These
% are some common ones, but not all are needed for this document. There is no
% real harm loading your favorites out of habit. 


\usepackage{algorithm,algorithmic,alltt,amsmath,amssymb,bm,
    cancel,color,fullpage,graphicx,listings,mathrsfs,
    multirow,setspace,subcaption,upgreek,xcolor}
\usepackage[numbered,framed]{matlab-prettifier}
\usepackage[colorlinks]{hyperref}
\usepackage[nameinlink,noabbrev]{cleveref}
\usepackage[verbose]{placeins}
\usepackage{caption}
\usepackage[skip=0.1ex, belowskip=1ex,
            labelformat=brace,
            singlelinecheck=off]{subcaption}

\doublespacing

%%%%%%%%%%%%%%%%%%%%%%%%%%%% Operators %%%%%%%%%%%%%%%%%%%%%%%%%%%%%%%
% Your personal shortcuts. You do not need to use any. argmax and argmin
\DeclareMathOperator*{\argmax}{arg\,max}
\DeclareMathOperator*{\argmin}{arg\,min}

%% Distributions
\newcommand{\N}{\mathcal{N}}
\newcommand{\U}{\mathcal{U}}
\newcommand{\Poi}{{\text Poisson}}
\newcommand{\Exp}{{\text Exp}}
\newcommand{\G}{\mathcal{G}}
\newcommand{\Ber}{{\text Bern}}
\newcommand{\Lap}{{\text Laplace}}
\newcommand{\btheta}{\boldsymbol{\theta}}
\newcommand{\bSigma}{\boldsymbol{\Sigma}}
% \usepackage[left=1cm,right=2.5cm,top=2cm,bottom=1.5cm]{geometry}
% Code blocks formatting

\definecolor{MyDarkGreen}{rgb}{0.0,0.4,0.0}

% For faster processing, load Matlab syntax for listings
\lstloadlanguages{Matlab}%
\lstset{language=Matlab,        % Use MATLAB
        % frame=single,   % Single frame around code
        basicstyle=\small\ttfamily,     % Use small true type font
        keywordstyle=[1]\color{blue}\bfseries,  % MATLAB functions bold and blue
        keywordstyle=[2]\color{purple}, % MATLAB function arguments purple
        keywordstyle=[3]\color{blue}\underbar,  % User functions underlined and blue
        identifierstyle=,       % Nothing special about identifiers
        % Comments small dark green courier
        commentstyle=\usefont{T1}{pcr}{m}{sl}\color{MyDarkGreen}\small,
        stringstyle=\color{purple},     % Strings are purple
        showstringspaces=false,         % Don't put marks in string spaces
        tabsize=2,      % 2 spaces per tab
        %%% Put standard MATLAB functions not included in the default language here
        morekeywords={xlim,ylim,var,alpha,factorial,poissrnd,normpdf,normcdf},
        %%% Put MATLAB function parameters here
        morekeywords=[2]{on, off, interp},
        %%% Put user defined functions here
        morekeywords=[3]{hw1,hw2,},
        gobble=4,
        morecomment=[l][\color{blue}]{...},     % Line continuation (...) like blue comment
        numbers=left,   % Line numbers on left
        firstnumber=1,          % Line numbers start with line 1
        numberstyle=\tiny\color{blue},  % Line numbers are blue
        stepnumber=5    % Line numbers go in steps of 5
}

%% Probability
\newcommand{\E}[1]{\mathbb{E}[#1]}
\newcommand{\Cov}[2]{\mathbb{C}\mathrm{ov}(#1,#2)}

%% Bold font for vectors from Ernesto, but I do not know how the first one
%  works, but it seems necessary for the second?
\def\*#1{\mathbf{#1}}
\newcommand*{\V}[1]{\mathbf{#1}}

%%%%%%%%%%%%%%%%%%%%%%%%%%%%%%%%%%%%%%%%%%%%%%%%%%%%%%%%%%%%%%%%%%%%%%

\begin{document}


\centerline{\it CS 577}
\centerline{\it HW \#6 Submission}
\centerline{\it Name: Joses Omojola}

Questions from Part A-B were completed in matlab and saved in the \emph{hw6.m} program. The program creates an \emph{output} folder to save images, 
so that the root directory is not always cluttered. The program can be run using \textit{hw6()}, and the results to coding 
questions are printed in terminal.

\begin{enumerate}

    \item[Part-A1.]
    \ \\
    The light direction matrix was read into matlab using the \emph{readmatrix()} function. Seven photometric images were imported and flattened across 
    RGB channels to create a matrix (400,400,7). Two nested loops were used to iterate over each pixel in the image and the intensity across the 7 images 
    was saved in a $I$ matrix with dimensions (7,1). The normal matrix $N$ was derived from \autoref{eqn:Equation1}, where $N=\rho \times \hat{n}$.
    \begin{equation}
    \tag{1}
    I = \rho \times \hat{n} \cdot s
    \label{eqn:Equation1}
    \end{equation}
    Because more than 3 images were provided $s$ is not a square matrix, and the normal matrix $N$ had to calculated using the least squares estimation method 
    $N = (s^T s)^{-1} s^T I$. The resulting normal vector was converted into a unit vector $\mathbf{\hat{n}}$ by dividing the normal vector by the albedo 
    $\hat{n} = \frac{N}{\rho} = \frac{N}{\lVert N \rVert}$, providing solutions for both the \emph{image normals} and \emph{albedo}. The canonical image was 
    generated by assuming the light direction was fixed at $(0,0,1)$ and iterating over two nested loops to calculate the intensity at each pixel with 
    \autoref{eqn:Equation1}. The resulting image is documented in \autoref{fig:Figure1}.

    \begin{figure}[H]
        \centering
        \includegraphics[scale=0.5]{output/f1_canonical_view_of_images.png}
        \caption{Recovered canonical image of photometric stereos assuming that the surface is perpendicular to the camera direction (0,0,1).}
        \label{fig:Figure1}
    \end{figure}

    \FloatBarrier 

    \item[Part-A2.]
    \ \\
    The depth map of the surface was generated by integrating the partial derivatives of the normal matrix along all rows and columns. The origin was defined 
    at the top-left of the image and the resulting surface is shown in \autoref{fig:Figure2}.

    \begin{figure}[H]
        \centering
        \includegraphics[scale=0.5]{output/f2_surface_depth_map.png}
        \caption{3D depth map of image surface derived from integration of estimated normals view from (incl=107, azi=64).}
        \label{fig:Figure1}
    \end{figure}

    \FloatBarrier 

    The surface map can be estimated using 2 nested for loop to define its path but I used \emph{cumsum()} and \emph{repmat()} because they were faster and the 
    resulting output was similar.

    \item[Part-A3.]
    \ \\
    From the 3D surface, the white quadrants of the image have higher pixel values ~250, while darker quadrants in the top right and bottom left have lower pixel 
    values. When compared to the 3D surface, bright pixel quadrants have a maxima ~50 while dark quadrants have a minima of ~-50. This suggests that peaks are 
    associated with bright pixels and valleys are associated with dim pixels.

    \item[Part-A4.]
    \ \\
    The normal of the 3D surface was estimated using \emph{surfnorm()}, and the resulting normal matrix was used to recompute the canonical image with 
    \autoref{eqn:Equation1}, assuming the camera position was still fixed at (0,0,1). The resulting surface normals had to be rotated to aligned it with the orientation 
    of the image. The main result is shown in \autoref{fig:Figure3}, and different light directions were tested to replicate the patterns observed in the input 7 images.

    \begin{figure}[H]
        \centering
        \includegraphics[scale=0.5]{output/f3_reconstructed_image_from_surface_normals.png}
        \caption{Recovered canonical image of photometric stereos using normals from 3D surface, and assuming that the camera direction is fixed at (0,0,1).}
        \label{fig:Figure3}
    \end{figure}

    \FloatBarrier 

    \item[Part-B1.]
    \ \\
    The depth map of the color photometric surfaces were recovered using the light direction and color intensity files across 2 nested loops to invert for the 
    image normals across each channel. The three normal vectors were averaged across each channel for the normal matrix. Because the albedo is expected to be 1 
    (pure white), the normal vector was not converted to a unit vector. The resulting matrix was integrated to obtain the depth maps, which are shown in 
    \autoref{fig:Figure4}.

    \begin{figure}[H]\centering
        \hspace*{-1.2in}
        \begin{subfigure}{0.40\textwidth}
            \includegraphics[scale=0.50]{output/f4_cps1_depth_map.png}
            \caption{Depth map for Photometric stereo 1}
            \label{fig:Figure4a}
        \end{subfigure}
    \hfil
        \begin{subfigure}{0.40\textwidth}
        \includegraphics[scale=0.50]{output/f5_cps2_depth_map.png}
        \caption{Depth map for Photometric stereo 2}
        \label{fig:Figure4b}
        \end{subfigure}
        \caption{Depth maps using recovered from surface normals}
        \label{fig:Figure4}
    \end{figure}


\end{enumerate}

\end{document}